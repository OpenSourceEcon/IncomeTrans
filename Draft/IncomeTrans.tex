\documentclass[letterpaper,12pt]{article}

\usepackage{threeparttable}
\usepackage{geometry}
\geometry{letterpaper,tmargin=1in,bmargin=1in,lmargin=1.25in,rmargin=1.25in}
\usepackage[format=hang,font=normalsize,labelfont=bf]{caption}
\usepackage{amsmath}
\usepackage{multirow}
\usepackage{array}
\usepackage{delarray}
\usepackage{amssymb}
\usepackage{amsthm}
\usepackage{lscape}
\usepackage{natbib}
\usepackage{setspace}
\usepackage{float,color}
\usepackage[pdftex]{graphicx}
\usepackage{pdfsync}
\usepackage{verbatim}
\usepackage{placeins}
\usepackage{geometry}
\usepackage{pdflscape}
\synctex=1
\usepackage{hyperref}
\hypersetup{colorlinks,linkcolor=red,urlcolor=blue,citecolor=red}
\usepackage{bm}


\theoremstyle{definition}
\newtheorem{theorem}{Theorem}
\newtheorem{acknowledgement}[theorem]{Acknowledgement}
\newtheorem{algorithm}[theorem]{Algorithm}
\newtheorem{axiom}[theorem]{Axiom}
\newtheorem{case}[theorem]{Case}
\newtheorem{claim}[theorem]{Claim}
\newtheorem{conclusion}[theorem]{Conclusion}
\newtheorem{condition}[theorem]{Condition}
\newtheorem{conjecture}[theorem]{Conjecture}
\newtheorem{corollary}[theorem]{Corollary}
\newtheorem{criterion}[theorem]{Criterion}
\newtheorem{definition}{Definition} % Number definitions on their own
\newtheorem{derivation}{Derivation} % Number derivations on their own
\newtheorem{example}[theorem]{Example}
\newtheorem{exercise}[theorem]{Exercise}
\newtheorem{lemma}[theorem]{Lemma}
\newtheorem{notation}[theorem]{Notation}
\newtheorem{problem}[theorem]{Problem}
\newtheorem{proposition}{Proposition} % Number propositions on their own
\newtheorem{remark}[theorem]{Remark}
\newtheorem{solution}[theorem]{Solution}
\newtheorem{summary}[theorem]{Summary}
\bibliographystyle{aer}
\newcommand\ve{\varepsilon}
\renewcommand\theenumi{\roman{enumi}}
\newcommand\norm[1]{\left\lVert#1\right\rVert}

\begin{document}

\begin{titlepage}
\title{
  Downward Risk in Labor Income: Age, Gender, Education, Crime, and Inequality \thanks{Thanks to Fulin Guo for excellent research assistance. This research benefited from support from the Open Source Economics Laboratory at the University of Chicago. All Python code and documentation for the computational model is available at \href{https://github.com/OpenSourceEcon/IncomeTrans}{https://github.com/OpenSourceEcon/IncomeTrans} [currently a private repository].}
}
\author{
  Richard W. Evans\footnote{University of Chicago, Becker Friedman Institute, McGiffert House, Room 208, Chicago, IL 60637, (773) 702-9169, \href{mailto:rwevans@uchicago.edu}{rwevans@uchicago.edu}.}}
\date{April 2018 \\
  \scriptsize{(version 19.04.a)}}
\maketitle
\vspace{-9mm}
\begin{abstract}
\small{This paper uses the PSID to estimate empirical Markov transition matrices for each year of the survey for the probability of transitioning from one quartile of the labor income distribution to any of the other quartiles of the labor distribution. We merge in crime rate, high school dropout rate, and inequality level data associated with the geography of each PSID respondent because these variables have shown to be significant predictors of intergenerational mobility. We show that age, gender, and these location-specific factors also predict high frequency transition probabilities in labor income. We are currently applying for access to the zip code level data from the PSID, and we will also apply for access to the IRS panel of earnings transitions.
\vspace{3mm}

%\noindent\textit{keywords:}\: Static scoring, revenue estimates, dynamic scoring.
%
%\vspace{3mm}
%
\noindent\textit{JEL classification:} put codes here
}

\end{abstract}
\thispagestyle{empty}
\end{titlepage}


\begin{spacing}{1.5}

\section{Introduction}\label{SecIntro}

  Recent work has focused on estimating the stochastic income processes faced by different households. The accuracy of these processes are an essential input to modeling household behavior. The expected riskiness of labor income varies by earner personal and location characteristics. Precautionary savings and labor supply behavior is influenced by the perceived riskiness of future income streams. This paper describes how downward income risk changes based on individual characteristics (age and gender) as well as some location specific characteristics shown to be important in the literature (crime rate, dropout rate, and inequality).

  \citet{GuvenenEtAl:2014} study how labor earnings transition probabilities change over the business cycle. Their main findings are that the skewness of the shock process shifts from expansion to recession. They also find that the earnings process for top earners is significantly different from that of the rest of the distribution and that the downside earnings risk for top earners increases disproportionately during a recession.

  \citet{DeBackerRamnath:2019} estimate these transition probabilities for narrow quantiles of the U.S. population using administrative panel data from the Internal Revenue Service.

%   Heterogeneous agent models have become the norm in macroeconomics. This development has added more richness and realism to macroeconomic models and allowed for the exploration of topics related to distributional issues that could not be otherwise addressed. However, there often remains a lack of detail in the way policy instruments are incorporated into dynamic general equilibrium models. The gap between the rich heterogeneity in model agents and lack of policy detail can be especially striking in the context of models used to evaluate tax policy. This gap is often due to the intractability of modeling the fine details of real-world policy.

%   \citet{KL2016} ask policy relevant questions regarding tax policy and have a rich model comprised of agents with heterogeneous skill-levels, assets, and age. But they model the tax code using linear tax functions. Even models at the frontier of the dynamic of analysis of fiscal policy, such as \citet{Nishiyama2015}, impose tax functions that are progressive but do not allow for marginal rates on a particular income source to be a function of other income.

%   Our contributions in this paper are primarily methodological. First, we propose a flexible functional form for tax rates that has the smoothness and monotonicity properties necessary for solving a DGE model while retaining much of the heterogeneity found in microsimulation model tax data. The tax functions that we propose can capture progressive rates, include negative tax rates, and account for the influence of income across sources on marginal rates. That is, our tax functions are multivariate, like the income tax code in the U.S. and many other countries, where income from one source affects marginal rates from other sources of income. Second, we describe a methodology where one can easily fit these tax functions using the output of a microsimulation model. The use of a microsimulation model is important in that these models are able to capture the rich detail of tax policy and how it affects households with different economic and demographic characteristics. The tax functions we propose then map the results of the microsimulation model, the computed average and marginal tax rates, into parameterized functions that can be used in a macroeconomic model. We tailor our functions here to a specific microsimulation model and DGE model, but the methodology we propose can be scaled up or down to account for models with more or less heterogeneity.

%   Our approach has two distinct advantages. It allows the DGE model to capture more detail of tax policy in than previously used methods. It also greatly reduces the cost to incorporating rich policy detail and counterfactuals into macroeconomic analysis. The bridge we build between the microsimulation model and the macroeconomic model essentially automates this process.

%   Others have used parameterized tax functions to represent the tax code in general equilibrium models. \citet{FR1993} estimate tax rate functions that vary by lifetime income group and age, but their marginal and average rates are not functions of realized income. \citet{DZ2013} follow a similar methodology.  Many of these studies use micro data to estimate the tax functions. For example, \citet{FR1993} use the Panel Study for Income Dynamics to estimate ordinary least squares models that identify the parameters of their tax functions. \citet{GKV2014} use data from the Statistics of Income (SOI) Public Use File to calibrate average and marginal tax rate functions for various definitions of household income, separately for those with different household structures.  Other examples of the estimations of  flexible tax functions on labor or household income (in the U.S. and across other countries) come from \citet{GS1994}, \citet{GKO2014}, and \citet{HKS2014}.  \citet{Nishiyama2015} uses a version of the \citet{GS1994} tax function, but does not condition tax functions on age nor does he allow marginal tax rates to be multivariate functions of the agents' different income sources. Rather, the marginal tax rate on labor income is only a function of labor income and the marginal tax rate on capital income is constant.  \citet{Nishiyama2015} uses ordinary least squares to estimate the parameters of his proposed tax functions from data produced by the Congressional Budget Offices' microsimulation model.

%   Our approach uses the output of a microsimulation model, \texttt{Tax-Calculator}, to estimate effective and marginal tax rate functions that jointly vary by age, labor income, and capital income. This study is the first to incorporate this level of detail into the tax functions used in a DGE model.  It is also novel in the integration between the microsimulation and DGE models. Such integration not only allows one to estimate tax functions for current law policy, but also to estimate the parameters of tax functions that specify counterfactual tax policies---even those that adjust tax policy levers that are difficult to model explicitly in a general equilibrium framework.

%   In this paper, we apply our methodology by analyzing the macroeconomics effects of a simple change in tax policy---a 10-percent reduction in all statutory marginal tax rates on personal income and a doubling of the standard deduction for each filer type. For the purposes of our simulations, we assume these changes in tax law are permanent and are instituted on January 1, 2017 with no anticipatory effects. This policy experiment is a canonical example, used by \citet{CBO2004} and \citet{DM2011} to show the effects of various modeling approaches to dynamic analysis of tax policy. In our experiment, we add the increase in the standard deduction to our policy experiment in order to include a change in the tax code that differentially affects marginal and average tax rates, which further exemplifies the power of our method of incorporating microsimulation tax data.

%   The paper is organized as follows. Section \ref{SecMacroModel} provides a brief overview of how taxes enter a dynamic general equilibrium model.\footnote{More detail on the DGE model is available in the Appendix.} We then describe the functional form for the tax functions we use and describe how they map to the DGE model in Section \ref{SecTaxFunctions}. Section \ref{SecIntegr} then details how the parameters of these tax functions are estimated from the output of a microsimulation model. In Section \ref{sec:compare_fit} we show how our tax functions compare to others.  We present an illustrative example of our methodology in Section \ref{SecResults}, using a canonical tax policy reform. Section \ref{sec:conclude} concludes.


% \section{Taxes in a DGE Model}\label{SecMacroModel}

%   To illustrate how tax policy enters a macroeconomic dynamic general equilibrium model, we use the overlapping generations model of \citet{DEMPRW2015}. This model has some theoretical simplifications, such as deterministic lifetime earnings ability profiles and a government budget constraint that is balanced every period. Our focus here is independent of the particular DGE model. We are demonstrating how the richness of tax policy can be tractably integrated into any DGE macroeconomic model. We provide a full description of the model in Appendix \ref{AppDGE}, but here we focus specifically on the details of the model that are relevant for describing how and where taxes enter the DGE model. In particular, we describe the dimensions of heterogeneity in the model and how income taxes affect household decisions.

%   In \citet{DEMPRW2015}, agents are heterogeneous in age, their lifetime labor productivity profiles, and the wealth the model households accumulate (which is endogenous). In particular, there are seven lifetime income groups in the model. Income is endogenous, so lifetime income groups are defined by potential earnings and the earnings profiles we estimate are over hourly earnings.\footnote{Our methodology to define and estimate these earnings profiles follows \citet{FR1993} and is described in detail in \citet{DEMPRW2015}.} The estimated earnings profiles are shown in Figure \ref{FigLogAbility_text}.

%   \begin{figure}[htb]\centering \captionsetup{width=4.0in}
%     \caption{\label{FigLogAbility_text}\textbf{Exogenous life cycle income ability paths $\log(e_{j,s})$ with $S=80$ and $J=7$}}
%     \fbox{\resizebox{4.0in}{2.7in}{\includegraphics{./images/ability_log_2D.png}}}
%   \end{figure}

%   Model agents are economically active for as many as $S$ years, facing mortality risk that is a function of their age, $s$.  Lifetime income groups are noted with the subscript $j$ and the effective labor units (productivity) over the lifecycle for each type is given by $e_{j,s}$. The model year is denoted by the subscript $t$.  Model agents choose consumption, $\hat{c}_{j,s,t}$, savings, $\hat{b}_{j,s,t}$, and labor supply, $\hat{n}_{j,s,t}$.  The hats over the variables denote that they have been stationarized (see Appendix \ref{AppDGE} for more details).

%   Our focus is on individual income taxes. The effect of such taxes on model agents' decisions is captured in three equations. First, the total income tax paid by the model agent determines after-tax resources available for consumption and savings. This is related through the budget constraint, shown in Equation \ref{EqBC_text}:

%   \begin{equation}\label{EqBC_text}
%     \begin{split}
%       & \hat{c}_{j,s,t} = \left(1 + r_t\right)\hat{b}_{j,s,t} + \hat{w}_t e_{j,s}n_{j,s,t} + \frac{\hat{BQ}_{j,t}}{\lambda_j} - e^{g_y}\hat{b}_{j,s+1,t+1} - \hat{T}^{I}_{s,t} + \hat{T}^{H}_{t} \\
%     \end{split}
%   \end{equation}
%   where $r_t$ is the real interest rate at time $t$ and $\hat{w}_{t}$ is the stationarized wage rate at time $t$.  The parameter $\lambda_j$ is the fraction of the total working households of type $j$ in period $t$ and term $BQ_{j,t}$ represents total bequests from households in income group $j$ who died at the end of period $t-1$. The growth rate in labor augmenting technological change is given by $g_{y}$.  $\hat{T}^{I}_{s,t}$ is a function representing income and payroll taxes paid, which we specify more fully below in equation \eqref{EqNetTaxLiab_text}. $\hat{T}^{H}_{t}$ is a lump sum transfer given to all households, which does not vary with household tax liabilities.\footnote{In principle, one could specify the government transfer function in much the same way we specify the tax function.  However, given how benefits are administered in the U.S. (in particular, how benefits schedules vary from state to state), it is difficult to construct a calculator that determines benefits receipt for individuals under different policy options.  This is something we hope to address in future research, but that is beyond the scope of this paper's focus on tax policy.}

%   The income tax liability function $T^{I}_{s,t}$ is represented by an effective (i.e., average) tax rate function times income. Let $x\equiv \hat{w}_t e_{j,s}n_{j,s,t}$ represent stationary labor income, and let $y\equiv  r_t\hat{b}_{j,s,t}$ represent stationary capital income.  Income tax liability is given as:
%   \begin{equation}\label{EqNetTaxLiab_text}
%     \begin{split}
%       T^{I}_{s,t}(x, y) &= \tau_{s,t}(x, y)\bigl(x + y\bigr) \\
%     \end{split}
%   \end{equation}
%   Note that the both the tax liability function $T^{I}_{s,t}(x,y)$ and the effective tax rate function $\tau_{s,t}(x,y)$ are functions of stationarized labor income, $x$, and capital income, $y$, jointly. We detail the parametric specification of the effective tax rate functions $ETR_{s,t}(x,y)\equiv \tau_{s,t}(x,y)$ in Section \ref{SecTaxFunctions}.

%   The effects of marginal tax rates effects on consumption, savings, and labor supply can be seen in the necessary conditions characterizing the agent's optimal choices of labor supply and savings. The first order condition for the choice of labor is given by:
%   \begin{equation}\label{EqEulerLabStat_text}
%     \begin{split}
%       &(\hat{c}_{j,s,t})^{-\sigma}\Biggl(\hat{w}_t e_{j,s} - \frac{\partial\hat{T}^{I}_{s,t}}{\partial n_{j,s,t}}\Biggr) = \chi^n_{s}\biggl(\frac{b}{\tilde{l}}\biggr)\biggl(\frac{n_{j,s,t}}{\tilde{l}}\biggr)^{\upsilon-1}\Biggl[1 - \biggl(\frac{n_{j,s,t}}{\tilde{l}}\biggr)^\upsilon\Biggr]^{\frac{1-\upsilon}{\upsilon}} \\
%       &\qquad\qquad\qquad\qquad\qquad\qquad\qquad\qquad\forall j,t, \quad\text{and}\quad E+1\leq s\leq E+S \\
%       &\qquad\text{where}\quad  \hat{b}_{j,E+1,t} = 0 \quad\forall j,t
%     \end{split}
%   \end{equation}
%   The left hand side of the equation gives the marginal benefits to additional labor supply while the right hand side relates the marginal costs from the disutility of labor supply.  The parameter $\sigma$ is the coefficient of relative risk aversion from the constant relative risk aversion utility function.  $\chi^{n}_{s}$ are age-dependent utility weights on the disutility of labor supply, $\tilde{l}$ is the maximum hours an agent can work, and the parameters $b$ and $\upsilon$ are parameters of the disutility of labor function.\footnote{\citet{EvansPhillips:2016} detail how an elliptical functional form can closely approximate the marginal utilities of the more common constant Frisch elasticity disutility of labor function while also providing Inada conditions at both the upper and lower bounds of labor supply.}

%   Taxes affect the labor-leisure decision in \eqref{EqEulerLabStat_text} through the partial derivative, $\frac{\partial\hat{T}^{I}_{s,t}}{\partial n_{j,s,t}}$, or the change in tax liability from a change in labor supply.  We can decompose this marginal effect in the following way:
%   \begin{equation}\label{MTRx_derive}
%     \begin{split}
%       \frac{\partial\hat{T}^{I}_{s,t}}{\partial n_{j,s,t}} & = \frac{\partial\hat{T}^{I}_{s,t}}{\partial \hat{w}_{t}e_{j,s}n_{j,s,t}}\frac{\partial \hat{w}_{t}e_{j,s}n_{j,s,t}}{\partial n_{j,s,t}} \\
%       &= \frac{\partial\hat{T}^{I}_{s,t}}{\partial \hat{w}_{t}e_{j,s}n_{j,s,t}}\hat{w}_{t}e_{j,s}
%     \end{split}
%   \end{equation}
%   Let $x\equiv \hat{w}_t e_{j,s}n_{j,s,t}$ and $y\equiv  r_t\hat{b}_{j,s,t}$, we define the function describing the marginal tax rate on labor income as $MTRx_{s,t}(x,y)\equiv \frac{\partial\hat{T}^{I}_{s,t}}{\partial x}$.

%   The first order condition for the optimal lifetime savings decisions of an agent are given by the following dynamic Euler equation:
%   \begin{equation}\label{EqEulerSavStat_text}
%     \begin{split}
%       &(\hat{c}_{j,s,t})^{-\sigma} = ... \\
%       &e^{-g_y\sigma}\Biggl(\rho_s\chi^b_j \bigl(\hat{b}_{j,s+1,t+1}\bigr)^{-\sigma} + \beta(1-\rho_s)(\hat{c}_{j,s+1,t+1})^{-\sigma}\Biggl[1 + r_{t+1} - \frac{\partial \hat{T}^{I}_{s+1,t+1}}{\partial \hat{b}_{j,s+1,t+1}}\Biggr]\Biggr) \\
%       &\qquad\qquad\qquad\qquad\qquad\qquad\qquad\qquad\forall j,t,\quad\text{and}\quad E+1\leq s \leq E+S-1
%     \end{split}
%   \end{equation}
%   This condition states that, at an optimum, the marginal utility of consumption must be equation to the benefits from saving. These benefits on the right hand side of Equation \eqref{EqEulerSavStat_text} are given by the utility from accidental bequests and the discounted expected value of future consumption.  The parameter $\chi^{b}_{j}$ is the utility weight on bequests and $\rho_{s}$ is the age-dependent mortality rate. The parameter $\beta$ reflects the agents' rate of time preference.

%   Taxes affect savings through the partial $\frac{\partial \hat{T}^{I}_{s+1,t+1}}{\partial \hat{b}_{j,s+1,t+1}}$, which reflects the additional taxes paid as a function of an additional dollar of savings.  As we did with the change in taxes for a change in labor supply, we can decompose this as:
%   \begin{equation}\label{MTRy_derive}
%     \begin{split}
%       \frac{\partial \hat{T}^{I}_{s+1,t+1}}{\partial \hat{b}_{j,s+1,t+1}} & = \frac{\partial \hat{T}_{s+1,t+1}}{\partial  r_{t}\hat{b}_{j,s+1,t+1}}\frac{\partial \hat{r}_{t}\hat{b}_{j,s+1,t+1}}{\partial \hat{b}_{j,s+1,t+1}} \\
%       &= \frac{\partial\hat{T}^{I}_{s,t}}{\partial r_{t}\hat{b}_{j,s+1,t+1}}r_{t}
%     \end{split}
%   \end{equation}
%   Let $x\equiv \hat{w}_t e_{j,s}n_{j,s,t}$ and $y\equiv  r_t\hat{b}_{j,s,t}$, we define the function describing the marginal tax rate on capital income as $MTRy_{s,t}(x,y)\equiv \frac{\partial\hat{T}^{I}_{s,t}}{\partial y}$.

%   Any DGE model that incorporates individual income taxes will have some analogue to $ETR_{s,t}(x,y)$, $MTRx_{s,t}(x,y)$, and $MTRy_{s,t}(x,y)$. These tax concepts, average and marginal rates, will enter the model in the same general way as we describe above. It is these functions that we are estimating directly from microsimulation tax data. They will vary by age $s$ and time period $t$ and will each be functions of both labor income $x$ and capital income $y$. One of the contributions of this paper is the use of tax rate functions that vary with both labor and capital income. As will be shown in the next section, this characteristic seems to be an important feature of the tax data. With these definitions of the effective and marginal tax rate functions, we now turn to our parameterized functional form.


% \section{Tax Functions}\label{SecTaxFunctions}

%   Figure \ref{FigMicroTaxData3D} shows scatter plots of effective tax rates (ETR), marginal tax rates on labor income (MTRx), marginal tax rates on capital income (MTRy), and a histogram of the data points from the \texttt{Tax-Calculator} microsimulation model, each plotted as a function of labor income and capital income for all 42-year-olds in the year 2017. The data we use in the \texttt{Tax-Calculator} come from the 2009 IRS Public Use File and a statistical match of the Current Population Survey (CPS) demographic data.\footnote{We discuss this microsimulation model and the data further in Section \ref{SecIntegr}.}  Labor and capital income are truncated at \$600,000 in order to more clearly see the shape of the data in spite of the long right tail of the income distribution. Although there is noise in the data, effective tax rates are generally increasing in both labor and capital income at a decreasing rate from some slightly negative level to an asymptote around 30 percent. This regular shape in effective tax rates is observed for all ages in all years of the budget window, 2017-2026.

%   \begin{figure}[htbp]\captionsetup{width=6.0in}
%     \caption{\label{FigMicroTaxData3D}\textbf{Scatter plot of ETR, MTRx, MTRy, and histogram as functions of labor income and capital income from microsimulation model: $t=2017$ and $s=42$ under current law}}
%     \fbox{\resizebox{6.0in}{5.0in}{\includegraphics{./images/Age42_2017_scatters.png}}}
%     \scriptsize{$^*$Note: Axes in the histogram in the lower-right panel have been switched relative to the other three figures in order to see the distribution more clearly.}
%   \end{figure}

%   Because of the regularity in the shape of the effective tax rates, we choose to fit a smooth functional form to these data that is able to parsimoniously fit this shape while also being flexible enough to adjust to a wide range of tax policy changes. Our functional form, shown in \eqref{EqETR} for the effective tax rate, is a Cobb-Douglas aggregator of two ratios of polynomials in labor and capital income. We use the same functional form for the effective and marginal tax rate functions. Important properties of this functional form are that it produces this bivariate negative exponential shape, is monotonically increasing in both labor income and capital income, and that it allows for negative tax rates. In order to capture variation in taxes by filer age and model year, we estimate functions for each model age and every year of the budget-window that the microsimulation model captures. In this way, we are able to map more of the heterogeneity from the microsimulation into the macro model than can be explicitly incorporated into a DGE model.

%   As an example, filing status is correlated with age and income.  Thus, although the DGE model we use does not explicitly account for filing status, we are able to capture some of the effects of filing status on tax rates by having age and income dependent functions for effective and marginal tax rates. As another example, investment portfolio decisions differ over the lifecycle and these are difficult to model in detail in a DGE model. By using age-dependent tax functions, we are able to capture some of the differentials in tax treatment across different assets (e.g. rates on dividends versus capital gains, tax-preferred retirement savings accounts, certain exemptions for interest income) even if the DGE model does not explicitly model these portfolio decisions.

%   Finally, consider that many macroeconomic models assume a single composite consumption good. Some of this composite good affects tax liability, such as the consumption of charitable contributions or housing. To the extent that the fraction of the composite good that comes from such consumption varies over a household's income and age, these tax functions will capture that, since they are fitted using microeconomic data that includes information on these tax-relevant forms of consumption.

%   Let $x$ be total labor income, $x\equiv \hat{w}_t e_{j,s}n_{j,s,t}$, and let $y$ be total capital income, $y\equiv  r_t\hat{b}_{j,s,t}$. We then write our tax rate functions as follows.
%   \begin{equation}\label{EqETR}
%     \begin{split}
%       \tau(x,y) = &\Bigl[\tau(x) + shift_x\Bigr]^\phi\Bigl[\tau(y) + shift_y\Bigr]^{1-\phi} + shift \\
%       &\text{where}\quad \tau(x) \equiv (max_x - min_x)\left(\frac{Ax^2 + Bx}{Ax^2 + Bx + 1}\right) + min_x \\
%       &\quad\text{and}\quad \tau(y) \equiv (max_y - min_y)\left(\frac{Cy^2 + Dy}{Cy^2 + Dy + 1}\right) + min_y \\
%       &\text{where}\quad A,B,C,D,max_x,max_y,shift_x,shift_y > 0 \quad\text{and}\quad\phi\in[0,1] \\
%       &\quad\text{and}\quad max_x > min_x \quad\text{and}\quad max_y > min_y
%     \end{split}
%   \end{equation}
%   Note that we let $\tau(x,y)$ represent the effective and marginal rate functions, $ETR(x,y)$, $MTRx(x,y)$ and $MTRy(x,y)$.  We assume the same functional form for each of these functions.  The parameters values will, in general, differ across the different functions (effective and marginal rate functions) and by age, $s$, and tax year, $t$.  We drop the subscripts for age and year from the above exposition for clarity.

%   By assuming each tax function takes the same form, we are breaking the analytical link between the the effective tax rate function and the marginal rate functions.  In particular, one could assume an effective tax rate function and then use the analytical derivative of that to find the marginal tax rate function.  However, we've found it useful to separately estimate the marginal and average rate functions.  One reason is that we want the tax functions to be able to capture policy changes that have differential effects on marginal and average rates.  For example, and relevant to the policy experiment we present below, a change in the standard deduction for tax payers would have a direct effect on their average tax rates.  But it will have secondary effect on marginal rates as well, as some filers will find themselves in different tax brackets after the policy change. These are smaller and second order effects. When tax functions are are fit to the new policy, in this case a lower standard deduction, we want them to be able to represent this differential impact on the marginal and average tax rates. The second reason is related to the first. As the additional flexibility allows us to model specific aspects of tax policy more closely, it also allows us to better fit the parameterized tax functions to the data.

%   The key building blocks of the functional form Equation \eqref{EqETR} are the $\tau(x)$ and $\tau(y)$ univariate functions. The ratio of polynomials in the $\tau(x)$ function $\frac{Ax^2 + Bx}{Ax^2 + Bx + 1}$ with positive coefficients $A,B>0$ and positive support for labor income $x>0$ creates a negative-exponential-shaped function that is bounded between 0 and 1, and the curvature is governed by the ratio of quadratic polynomials. The multiplicative scalar term $(max_x-min_x)$ on the ratio of polynomials and the addition of $min_x$ at the end of $\tau(x)$ expands the range of the univariate negative-exponential-shaped function to $\tau(x)\in[min_x, max_x]$. The $\tau(y)$ function is an analogous univariate negative-exponential-shaped function in capital income $y$, such that $\tau(y)\in[min_y,max_y]$.

%   The respective $shift_x$ and $shift_y$ parameters in Equation \eqref{EqETR} are analogous to the additive constants in a Stone-Geary utility function. These constants ensure that the two sums $\tau(x) + shift_x$ and $\tau(y) + shift_y$ are both strictly positive. They allow for negative tax rates in the $\tau(\cdot)$ functions despite the requirement that the arguments inside the brackets be strictly positive. The general $shift$ parameter outside of the Cobb-Douglas brackets can then shift the tax rate function so that it can accommodate negative tax rates. The Cobb-Douglas share parameter $\phi\in[0,1]$ controls the shape of the function between the two univariate functions $\tau(x)$ and $\tau(y)$.

%   This functional form for tax rates delivers flexible parametric functions that can fit the tax rate data shown in Figure \ref{FigMicroTaxData3D} as well as a wide variety of policy reforms. Further, these functional forms are monotonically increasing in both labor income $x$ and capital income $y$. This characteristic of monotonicity in $x$ and $y$ is essential for guaranteeing convex budget sets and thus uniqueness of solutions to the household Euler equations. The assumption of monotonicity does not appear to be a strong one when viewing the the tax rate data shown in Figure \ref{FigMicroTaxData3D}. While it does limit the potential tax systems to which one could apply our methodology, tax policies that do not satisfy this assumption would result in non-convex budget sets and thus require non-standard DGE model solutions methods and would not guarantee a unique equilibrium. The 12 parameters of our tax rate functional form from \eqref{EqETR} are summarized in Table \ref{TabTaxParams}.

%   \begin{table}[htbp] \centering \captionsetup{width=5.0in}
%   \caption{\label{TabTaxParams}\textbf{Description of tax rate function $\tau(x,y)$ parameters}}
%     \begin{threeparttable}
%     \begin{tabular}{>{\footnotesize}c |>{\footnotesize}l }
%       \hline\hline
%       Symbol & \quad\quad\quad\quad Description  \\
%       \hline
%       $A$ & Coefficient on squared labor income term $x^2$ in $\tau(x)$ \\
%       $B$ & Coefficient on labor income term $x$ in $\tau(x)$ \\
%       $C$ & Coefficient on squared capital income term $y^2$ in $\tau(y)$ \\
%       $D$ & Coefficient on capital income term $y$ in $\tau(y)$ \\
%       $max_x$ & Maximum tax rate on labor income $x$ given $y=0$  \\
%       $min_x$ & Minimum tax rate on labor income $x$ given $y=0$ \\
%       $max_y$ & Maximum tax rate on capital income $y$ given $x=0$ \\
%       $min_y$ & Minimum tax rate on capital income $y$ given $x=0$ \\
%       $shift_x$ & shifter $>|min_x|$ ensures that $\tau(x) + shift_x > 0$ despite potentially \\
%       & \quad negative values for $\tau(x)$ \\
%       $shift_y$ & shifter $>|min_y|$ ensures that $\tau(y) + shift_y > 0$  despite potentially \\
%       & \quad negative values for $\tau(y)$ \\
%       $shift$ & shifter (can be negative) allows for support of $\tau(x,y)$ to include \\
%       & \quad negative tax rates \\
%       $\phi$ & Cobb-Douglas share parameter between 0 and 1 \\
%       \hline\hline
%     \end{tabular}
%     % \begin{tablenotes}
%     %   \scriptsize{\item[]Note: Maybe put sources here.}
%     % \end{tablenotes}
%     \end{threeparttable}
%   \end{table}

%   Figure \ref{FigMicroTaxEst3D} shows the estimated function surfaces for tax rate functions for the effective tax rate (ETR), marginal tax rate on labor income (MTRx), and marginal tax rate on capital income (MTRy) data shown in Figure \ref{FigMicroTaxData3D} for age $s=42$ individuals in period $t=2017$ under the current law. Section \ref{SecEstTaxFunc} details the nonlinear weighted least squares estimation method of the 12 parameters in Table \ref{TabTaxParams}, but before we detail those methods we show here that the functional form is able to fit the data closely. And the estimated parameters and the corresponding function surface change whenever any of the many policy levers in the microsimulation model that generate the tax rate data are adjusted. The total tax liability function is simply the effective tax rate function times total income $\tau(x,y)(x+y)$.

%   \begin{equation}\label{EqTotTaxLiab}
%     \begin{split}
%       T^{I}_{s,t}(x,y) &\equiv ETR_{s,t}(x,y)\bigl(x + y\bigr) \\
%       &= \biggl(\Bigl[\tau_{s,t}(x) + shift_{x,s,t}\Bigr]^\phi_{s,t}\Bigl[\tau_{s,t}(y) + shift_{y,s,t}\Bigr]^{1-\phi_{s,t}} + shift_{s,t}\biggr)\bigl(x + y\bigr)
%     \end{split}
%   \end{equation}

%   \begin{figure}[htbp]\captionsetup{width=6.0in}
%     \caption{\label{FigMicroTaxEst3D}\textbf{Estimated tax rate functions of ETR, MTRx, MTRy, and histogram as functions of labor income and capital income from microsimulation model: $t=2017$ and $s=42$ under current law}}
%     \fbox{\resizebox{6.0in}{5.0in}{\includegraphics{./images/Age42_2017_vsPred.png}}}
%     \scriptsize{$^*$Note: Axes in the histogram in the lower-right panel have been switched relative to the other three figures in order to see the distribution.}
%   \end{figure}

%   As we describe above, each rate function ($ETR,MTRx,MTRy$) varies by age, $s$, and tax year $t$. This means a large number of parameters must be estimated. In particular, using our illustrative example with the model of \citet{DEMPRW2015}, we will need to fit 12 parameters for each of three tax rate functions for each age (21 to 100) during each of the 10 years of the budget window with the estimated functions in the last year of the budget window assumed to be permanent. The microsimulation model we use, \texttt{Tax-Calculator} is able to provide marginal and average tax rates for 10-years forward from the present.\footnote{This is the standard timeframe considered by policy analysts analyzing the effects of tax policy on the federal budget.}  The DGE model is solved from the current period forward through the steady-state.  The steady-state is generally arrived at well beyond a time horizon of 10 years.  Thus we allow variation the in the rate functions only over this 10-year budget window and fix the parameters of the rate functions to the last year of the window for years $t\geq10$.  Thus, in our illustrative example, there are 2,400 tax rate functions comprised of 28,800 parameters.

%   Because we allow these many functions of labor income and capital income to be independently estimated for each tax rate type, age, and year, we can capture many of the characteristics and discrete variation in the tax code while still preserving the smoothness and monotonicity of the tax functions within each type, age, and year. This monotonicity and smoothness is sufficient to guarantee uniqueness and tractability of the computational solution of the household Euler equations. Allowing for different tax rate functions by age and time period also implicitly incorporates heterogeneity in the data in dimensions that we cannot model in the DGE model, such as broader income items, deductions items, credits, and filing unit structure. The effect of such heterogeneity on tax burdens will affect the effective tax rate functions we fit to the output of the microsimulation model.

%   \begin{table}[htbp] \centering \captionsetup{width=3.3in}
%   \caption{\label{TabPhiValues}\textbf{Average values of $\phi$ for ETR, MTRx, and MTRy for age bins in period $t=2017$}}
%     \begin{threeparttable}
%     \begin{tabular}{>{\footnotesize}l |>{\footnotesize}c >{\footnotesize}c >{\footnotesize}c |>{\footnotesize}c}
%       \hline\hline
%       & \multicolumn{3}{c}{\footnotesize{Age ranges}} & \\
%       & 21 to 54 & 55 to 65 & 66 to 80 & All years \\
%       \hline
%     $ETR$ & 0.66  & 0.28  & 0.38  & 0.44 \\
%     $MTRx$ & 0.89  & 0.31  & 0.23  & 0.48 \\
%     $MTRy$ & 0.77  & 0.25  & 0.14  & 0.43 \\
%       \hline\hline
%     \end{tabular}
%     \begin{tablenotes}
%       \scriptsize{\item[*]Note: Even though agents in the OG model live until age 100, the tax data was too sparse to estimate functions for ages greater than 80. For ages 81 to 100, we simply assumed the age 80 estimated tax functional forms.}
%     \end{tablenotes}
%     \end{threeparttable}
%   \end{table}


%   It is difficult to show all the estimated tax functions for every age and period in the budget window. But Table \ref{TabPhiValues} gives a description of the estimated values of the $\phi$ parameter. This parameter $\phi$ in the tax function \eqref{EqETR} governs how important the interaction is between labor income and capital income for determining tax rates. The further interior is $\phi$ (away from 0 or 1), the more important it is to model tax rates as functions of both labor income and capital income. And the closer $\phi$ is to 1, the more important is labor income for determining tax rates.

%   Two key results jump out from Table \ref{TabPhiValues}. First, it is clear that the interaction between labor income and capital income is significant at all ages for determining effective tax rates $ETR$, marginal tax rates on labor income $MTRx$, and marginal tax rates on capital income $MTRx$. The last column of Table \ref{TabPhiValues} shows the average $\phi$ value for all ages in the data to be around 0.45 for all three tax rate types. This suggests that models that use univariate tax functions of any type of income miss important information and incentives present in the tax code.

%   A second result from Table \ref{TabPhiValues} is that the relative importance of labor income in determining tax rates varies over the life cycle in similar ways for each tax rate type ($ETR$, $MTRx$, and $MTRy$). The first three columns of each row of Table \ref{TabPhiValues} show that labor income is most important for determining tax rates between the ages of 21 and 54 and that capital income is most important for determining tax rates between the ages of 55 and 65. For marginal tax rates capital income continues to be the most important determinant after age 65, but capital income and labor income are equally important determinants of the effective tax rate $ETR$ after age 65. This suggests that models that use tax functions that do not vary with age also miss some important information and incentives present in the tax code.


% \section{Integration of microsimulation model with DGE model}\label{SecIntegr}

%   An important part of the methodology we propose is our integration of tax functions estimated from the output of a microsimulation model into a DGE model.  The nature of DGE models is such that they cannot accommodate the degree of policy detail and filer heterogeneity that exist in the microdata. The analytics would be intractable and the computational burden too high. In addition, finding optimal solutions to the lifetime problem of each household would be extremely difficult due to the nonconvex optimization problem created by the kinks and cliffs in the current tax code or in the proposed policies.

%   In contrast, microsimulation models are perfectly suited to calculate the total taxes paid, effective tax rates, and marginal tax rates for a population with richly defined demographic heterogeneity.  Microsimulation models of tax policy also incorporate most of the detail in the tax code with respect to specific policy levers.  We fit smooth tax functions with the requisite properties to the tax rates determined through a microsimulation model. We then use those estimated parametric functions in a DGE macroeconomic model. In this way, we incorporate complexities of the actual tax code and their interactions with filer heterogeneity into a macroeconomic model, which is necessarily limited in terms of how much policy detail and household heterogeneity can be explicitly represented.


%   \subsection{Microsimulation model: \texttt{Tax-Calculator}}\label{SecMicrosim}

%     The microsimulation model we use is called \texttt{Tax-Calculator} and is maintained a group of economists, software developers, and policy analysts.\footnote{The documentation for using \texttt{Tax-Calculator} is available at \href{http://taxcalc.readthedocs.org/en/latest/index.html}{http://taxcalc.readthedocs.org/en/latest/index.html} A simple web application that provides an accessible user interface for \texttt{Tax-Calculator} is available from the Open Source Policy Center (OSPC) at \href{http://www.ospc.org/taxbrain/}{http://www.ospc.org/taxbrain/}. All the source code for the \texttt{Tax-Calculator} is freely available at \href{https://github.com/open-source-economics/Tax-Calculator}{https://github.com/open-source-economics/Tax-Calculator}.} Other than being completely open source, the \texttt{Tax-Calculator} is very similar to other tax calculators such as NBER's TaxSim and proprietary models used by think tanks and governmental organizations. For this reason, much of what we say below generalizes if one were to use those other microsimulation models. In this section, we outline the main structure of the \texttt{Tax-Calculator} microsimulation model, but encourage the interested reader to follow the links for more detailed documentation.

%     \texttt{Tax-Calculator} uses microdata on a sample of tax filers from the tax year 2009 Public Use File (PUF) produced by the IRS.\footnote{Technically, the \texttt{Tax-Calculator} could use other microdata as a source, but we choose to use the PUF for the relatively large sample size and the degree of detail provided for various income and deduction items.}  These data contain detailed records from the tax returns of about 200,000 tax filers who were selected from the population of filers through a stratified random sample of tax returns. These data come from IRS Form 1040 and a set of the associated forms and schedules. The PUF data are then matched to the Current Population Survey (CPS) to get imputed values for filer demographics such as age, which are not included in the PUF, and to incorporate households from the population of non-filers. The PUF-CPS match includes 219,815 filers.

%     Since these data are for calendar year 2009, they must be ``aged" to be representative of the potential tax paying population in the years of interest (e.g. the current year through the end of the budget window).  To do this, macroeconomic forecasts of wages, interest rates, GDP, and other variables are used to grow the 2009 values to be representative of the values one might see in subsequent years.  Adjustments to the weights applied to each observation in the microdata are also made.  More specifically, weights are adjusted to hit a number of targets in an optimization problem that sets out to minimize the distance between the extrapolated microdata values and the targets, with a penalty being applied for large changes in the weight individual observations from one year to the next.  The targets are comprised of a number of aggregate totals of line items from Form 1040 (and related Schedules) produced by SOI for the years 2010-2013.\footnote{For details on how these data are extrapolated, please see the \href{https://github.com/open-source-economics/taxdata}{\texttt{Tax Data}} program and associated documentation.}

%     Using these microdata, \texttt{Tax-Calculator} is able to determine total tax liability and marginal tax rates by computing the tax reporting that minimizes each filer's total tax liability given the filer's income and deductions items and the parameters describing tax law. The determination of total tax liability from the microsimulation model includes federal income taxes and payroll taxes but currently excludes state income taxes and estate taxes. The output of the microsimulation model includes forecasts of the total tax liability in each year, marginal tax rates in income sources, and items from the filers' tax returns for each of the 219,815 filers in the microdata.  To calculate marginal tax rates on any given income source, the model adds one cent to the income source for each filing unit in the microdata and then computes the change in tax liability.  The change in tax liability divided by the change in income (one cent) yields the marginal tax rate.  Population sampling weights are determined through the extrapolation and targeting of the microsimulation model. These weights allow one to calculate population representative results from the model. One can determine changes in tax liability and marginal tax rates across different tax policy options by doing the same simulation where the parameters describing the tax policy are updated to reflect the proposed policy rather than the baseline policy. The baseline policy used by \texttt{Tax-Calculator} is a current-law baseline.


%   \subsection{Mapping income from micro to macro model}

%     To map the output of the microsimulation model, which is based on income reported on tax returns, to the DGE model, where income is defined more broadly, we use the following definitions. In computing the effective tax rates from the microsimulation model, we divided total tax liability by a measure of ``adjusted total income".  Adjusted total income is defined as total income (Form 1040, line 22) plus tax-exempt interest income, IRA distributions, pension income, and Social Security benefits (Form 1040, lines 8b, 15a, 16a, and 20a, respectively).  We consider adjusted total income from the microsimulation model to be the counterpart of total income in the DGE model. Total income in the DGE model is the sum of capital and labor income.

%     We define labor income as earned income, which is the sum of wages and salaries (Form 1040, line 7) and self-employment income (Form 1040 lines 12 and 18) from the microdata. Capital income is defined as a residual.\footnote{This is not an ideal definition of capital income, since it includes transfers between filers (e.g., alimony payments) and from the government (e.g., unemployment insurance), but we have chosen this definition in order to ensure that all of total income is classified as either capital or labor income.}

%     To get the marginal tax rate on composite income amounts (e.g., labor income that is the sum of wage and self-employment income), we take a weighted average that accounts for negative income amounts.  In particular, to we calculate the weighted average marginal tax rate on composite of $n$ income sources as:

%     \begin{equation}
%       MTR_{composite} = \frac{\sum_{i=1}^{n} MTR_{n}*abs(Income_n)}{\sum_{i=1}^{n} abs(Income_n)}
%     \end{equation}

%     When we look at the raw output from the microsimulation model, we find that there are several observations with extreme values for their effective tax rate. Since this is a ratio, such outliers are possible, for example when the denominator, adjusted total income, is very small. We omit such outliers by making the following restrictions on the raw output of the microsimulation model. First, we exclude observations with an effective tax rate greater than 1.5 times the highest statutory marginal tax rate.  Second, we exclude observations where the effective tax rate is less than the lowest statutory marginal tax rate on income minus the maximum phase-in rate for the Earned Income Tax Credit (EITC).  Third, we drop observations with marginal tax rates in excess of 99\% or below the negative of the highest EITC rate (i.e., -45\% under current law).  These exclusions limit the influence of those with extreme values for their marginal tax rate, which are few and usually result from the income of the filer being right at a kink in the tax schedule.  Finally, since total income cannot be negative in the DGE model we use, we drop observations from the microsimulation model where adjusted total income is less than \$5.\footnote{We choose \$5 rather than \$0 to provided additional assurance that small income values are not driving large ETRs.}

%     Because the tax rates are estimated as functions of income levels in the microdata, we have to adjust the model income units to match the units of the microdata. To do this, we find the $factor$ such that $factor$ times average steady-state model income equals the mean income in the final year of the microdata.
%     \begin{equation}\label{EqIncFactor}
%       factor\sum_s \sum_j \bar{\omega}_s\lambda_j\left(\bar{w}e_{j,s}\bar{n}_{j,s} + \bar{r}\bar{b}_{j,s}\right) = \left(\text{data avg. income}\right)
%     \end{equation}
%     To be precise, the income levels in the model, $x$ and $y$, must be multiplied by this factor when they are used in the effective tax rate functions, marginal tax rate of labor income functions, and marginal tax rate of capital income functions of the form in Equation \eqref{EqETR}.


%   \subsection{Estimating tax functions}\label{SecEstTaxFunc}

%     With the output of the microsimulation model cleaned, we move to our estimation. We estimate a transformation of the $ETR$, $MTRx$, and $MTRy$ tax rate functions described in Equation \eqref{EqETR} for each age $s$ of the primary filer and time period $t$ in our data and budget window, respectively (2,400 separate specifications). That is, we estimate $\tau_{s,t}(x,y)$, $\frac{\partial T}{\partial x}\left(x,y\right)_{s,t}$, and $\frac{\partial T}{\partial y}\left(x,y\right)_{s,t}$. We transform these functions so that the labor income, $x$, and capital income, $y$, variables in the polynomials are transformed to percent deviations from their respective means. This helps with the scale of the variables in the optimization routine. The transformed ETR and MTR functions are estimated using a constrained, weighted, non-linear least squares estimator.  The weighting in this estimator come from the weights assigned to the filers in the microsimulation model.

%     Let $\bm{\theta}_{s,t}=(A,B,C,D,max_x,min_x,max_y,min_y,shift_x,shift_y,shift,\phi)$ be the full vector of 12 parameters of the tax function for a particular age of filers in a particular year. We first directly specify $min_x$ as the minimum tax rate in the data for age-$s$ and period-$t$ individuals for capital income close to 0 ($\$0<y<\$3,000$) and $min_y$ as the minimum tax rate for labor income close to 0 ($\$0<x<\$3,000$). We then set $shift_x = |min_x|+\ve$ and $shift_y = |min_y|+\ve$ so that the respective arguments in the brackets of \eqref{EqETR} are strictly positive. Let $\bar{\bm{\theta}}_{s,t}=\{min_x,min_y,shift_x,shift_y\}$ be the set of parameters we take directly from the data in this way.

%     We then estimate eight remaining parameters $\tilde{\bm{\theta}}_{s,t}=(A,B,C,D,max_x,max_y,shift,\phi)$ using the following nonlinear weighted least squares criterion,
%     \begin{equation}\label{EqThetaWSSQ}
%       \begin{split}
%         \bm{\hat{\theta}}_{s,t} = \tilde{\bm{\theta}}_{s,t}:\quad &\min_{\tilde{\bm{\theta}}_{s,t}}\sum_{i=1}^{N} \Bigl[\tau_{i}-\tau_{s,t}\bigl(x_i,y_i|\tilde{\bm{\theta}}_{s,t},\bar{\bm{\theta}}_{s,t}\bigr)\Bigr]^{2} w_i, \\
%         &\qquad\text{subject to}\quad A, B, C, D, max_x, max_y > 0, \\
%         &\qquad\text{and}\quad max_{x}\geq min_x, \quad\text{and}\quad max_{y}\geq min_y \quad\text{and}\quad \phi\in[0,1]
%       \end{split}
%     \end{equation}
%     where $\tau_{i}$ is the effective (or marginal) tax rate for observation $i$ from the microsimulation output, $\tau_{s,t}(x_i,y_i|\tilde{\bm{\theta}}_{s,t},\bar{\bm{\theta}}_{s,t})$ is the predicted average (or marginal) tax rate for filing-unit $i$ with $x_{i}$ labor income and $y_{i}$ capital income given parameters $\bm{\theta}_{s,t}$, and $w_{i}$ is the CPS sampling weight of this observation. The number $N$ is the total number of observations from the microsimulation output for age $s$ and year $t$. Figure \ref{FigMicroTaxEst3D} shows the typical fit of an estimated tax function $\tau_{s,t}\bigl(x,y|\hat{\bm{\theta}}_{s,t}\bigr)$ to the data. The data in Figure \ref{FigMicroTaxEst3D} are the same age $s=42$ and year $t=2017$ as the data Figure \ref{FigMicroTaxData3D}.

%     The underlying data can limit the number of tax functions that can be estimated.  For example, we use the age of the primary filer from the PUF-CPS match to be equivalent to the age of the DGE model household.  The DGE model we use allows for individuals up to age 100, however the data contain few primary filers with age above age 80.  Because we cannot reliably estimate tax functions for $s>80$, we apply the tax function estimates for 80 year-olds to those with model ages 81 to 100.  In the case certain ages below age 80 have too few observations to enable precise estimation of the model parameters, we use a linear interpolation method to find the values for those ages $21\leq s <80$ that cannot be precisely estimated.\footnote{We use two criterion to determine whether the function should be interpolated.  First, we require a minimum number of observations of filers of that age and in that tax year.  Second, we require that that sum of squared errors meet a pre-defined threshold.}


% \section{A Comparison of Tax Functions}\label{sec:compare_fit}

% In this section, we provide a comparison of the fit provided by our tax function specification.  We refer the reader to Sections \ref{SecTaxFunctions} for the description and justification of of our functional form.  We compare our functions to that of \citet{GS1994}, which remains one of the more flexible specifications in the literature, and one of the more widely used functional forms (e.g., see \citet{GKO2014} and \citet{Nishiyama2015}).  The \cite{GS1994} tax function is given by:

% \begin{equation}
% \label{eqn:GS}
% T = \varphi_{0}[I-(I^{-\varphi_{1}}+\varphi_{2})^{\frac{-1}{\varphi_{1}}}],
% \end{equation}

% \noindent\noindent where $T$ are total income taxes and $I=x+y$ is total income.  We transform this function to put it in terms of an effective tax rate:

% \begin{equation}
% \label{eqn:GS}
% ETR = \varphi_{0}[I-(I^{-\varphi_{1}}+\varphi_{2})^{\frac{-1}{\varphi_{1}}}]/I.
% \end{equation}

% In addition, we use our microdata from \texttt{Tax-Calculator} to estimate the above $ETR$ specification separately by tax year and age.  This is not done in others' work who use the \citet{GS1994} functional form, but this will give that specification the best chance to fit the data as closely as our preferred functional form proposed in this paper.  We use the same nonlinear least squares estimator to estimate the \citet{GS1994} functions.

% In Figure \ref{Compare_DEP_GS} we compare the \citet{GS1994} specification to ours.  In orange, we show our specification when total income is entirely made up of labor income and in green we show our specification when total income is comprised of 70\% labor income and 30\% capital income.  The \citet{GS1994} specification is in purple.  We can note a number of differences between these specifications from this picture.  First, our specification shows more ability to capture the negative average tax rates at the lower end of the income distribution.  Second, given the ability to have this negative intercept, our functional form can show more curvature over lower income ranges, allowing for a better fit to the steep gradient the data show over this range.  Finally, by comparing the orange and green lines, one can see the ability of the share parameter to account for the role capital income plays in lowering effective tax rates as total income increases.  In particular, our specification allows for the filers portfolio of income (i.e., the shares of total income derives from labor or capital income) at affect this average tax rate, which is a novel contribution of our functional form.

% \begin{figure}[htbp]\captionsetup{width=4.0in}
%   \centering
%   \caption{\label{Compare_DEP_GS}\textbf{Plot of estimated $ETR$ functions: $t=2017$ and $s=42$ under current law}}
%   \fbox{\resizebox{4.0in}{3.0in}{\includegraphics{./images/Compare_ETR_functions.png}}}
% \end{figure}

% To more precisely test the fit of these specification against each other, Table \ref{tab:DEP_GS_se} presents the standard errors of the estimates from these two specifications.  The table shows the over all fit and also splits these out by age.  What we find is that the ratio of polynomials function we propose shows a much better fit to the data, missing the effective tax rate by just under three percentage points on average, compared to an error of 5.5 percentage points in the estimated \citet{GS1994} model.

% \begin{table}[htbp] \centering \captionsetup{width=4.8in}
%   \caption{\label{tab:DEP_GS_se}\textbf{Standard error of the estimates of $ETR$ functions for age bins in period $t=2017$}}
%     \begin{threeparttable}
%     \begin{tabular}{>{\footnotesize}l |>{\footnotesize}c >{\footnotesize}c >{\footnotesize}c >{\footnotesize}c}
%       \hline\hline
%       & \multicolumn{4}{c}{\footnotesize{Age ranges}} \\
%       & All ages & 21 to 54 & 55 to 65 & 66 to 80 \\
%       \hline
%       Ratio of polynomials, $ETR$ & 2.98  & 3.42 & 1.67 & 2.22 \\
%       \citet{GS1994}, $ETR$ & 5.50  & 6.54  & 2.57  & 3.11 \\
%       \hline\hline
%     \end{tabular}
%     % \begin{tablenotes}
%     %   \scriptsize{\item[*]Add notes here.}
%     % \end{tablenotes}
%     \end{threeparttable}
%   \end{table}

% Note that we compare only the effective tax rate functions since \citet{GS1994} do not separately estimate marginal tax rate functions, as we do.  Instead, they derive the marginal tax rates analytically from their total tax function.  The implication, then, is that the marginal rates derives in this way will not fit the data as well as the effective tax rates, which were the target of the estimation.  As we note above, we eschewed this approach of analytically deriving the marginal tax rates in favor of separately estimating the parameters of the effective and marginal tax rate functions.  We've found this allows our model to better capture tax policy that differentially impacts average and marginal rates and to fit the data more closely.

% \section{A tax experiment}\label{SecResults}

%   The policy change we consider is a reduction in the statutory marginal rates on individual filers' ordinary income and an a doubling of the standard deduction for each tax filer type. The reduction in marginal rates is the same example used to illustrate the workings of overlapping generations models by \citet{CBO2004} and \citet{DM2011}. To this we add the change in the standard deduction to show how our methodology can capture policies that change average and marginal rates in distinct ways. The changes in marginal rates for each bracket are summarized in Table \ref{tab:rates}.  Table \ref{tab:std} shows the standard deduction by filer type under current law and under our policy experiment.

%   \begin{table}[htbp] \centering \captionsetup{width=3.0in}
%   \caption{\label{tab:rates}\textbf{Statutory marginal tax rates by bracket under the baseline current law versus policy change}}
%     \begin{threeparttable}
%     \begin{tabular}{>{\footnotesize}c |>{\footnotesize}c >{\footnotesize}c}
%       \hline\hline
%       Bracket & Baseline & Policy \\
%       \hline
%       1 & 0.100 & 0.090 \\
%       2 & 0.150 & 0.135 \\
%       3 & 0.250 & 0.225 \\
%       4 & 0.280 & 0.252 \\
%       5 & 0.330 & 0.297 \\
%       6 & 0.350 & 0.315 \\
%       7 & 0.396 & 0.356 \\
%       \hline\hline
%     \end{tabular}
%     \begin{tablenotes}
%       \scriptsize{\item[*]Note: The income cutoffs for each of these brackets vary by filer type.}
%     \end{tablenotes}
%     \end{threeparttable}
%   \end{table}

%   \begin{table}[htbp] \centering \captionsetup{width=3.0in}
%   \caption{\label{tab:std}\textbf{Standard deduction by filer type under the baseline current law versus policy change}}
%     \begin{threeparttable}
%     \begin{tabular}{>{\footnotesize}l |>{\footnotesize}r >{\footnotesize}r}
%       \hline\hline
%       \multicolumn{1}{c}{\footnotesize{Filing status}} & \multicolumn{1}{c}{\footnotesize{Baseline}} & \multicolumn{1}{c}{\footnotesize{Policy}} \\
%       \hline
%       Single                     &  \$6,350 & \$12,700 \\
%       Married, Filing Jointly    & \$12,700 & \$25,400 \\
%       Married, Filing Separately &  \$6,350 & \$12,700 \\
%       Head of Household          & \$9,350 & \$18,700 \\
%       Widow                      &  \$12,700 & \$25,400 \\
%       Dependent                  &  \$1,050 &  \$2,100 \\
%       \hline\hline
%     \end{tabular}
%     % \begin{tablenotes}
%     %   \scriptsize{\item[*]Note: The income cutoffs for each of these brackets vary by filer type.}
%     % \end{tablenotes}
%     \end{threeparttable}
%   \end{table}

%   Using the methodology described above, we use the microsimulation model to compute effective and marginal tax rates for each filing unit in the microdata under the baseline and policy tax law.  These data are then used in the estimation of the parameters of tax functions for each tax year, age, and under the baseline and reform policies.  We present the estimated $ETR$, $MTRx$, and $MTRy$ parametric functions in Table \ref{tab:tax_params} for age $s=42$ period $t=2017$ tax filers for both the baseline policy and the tax experiment. These are only 6 of the 3,600 tax functions that we estimate in the baseline and policy change combined.\footnote{Three types of tax functions times 10 year in budget window times 60 ages equals 1,800 functions. The rest of the estimated functions are available in two Python pickle files named \texttt{TxFuncEst\_baselineint.pkl} and \texttt{TxFuncEst\_policyint.pkl} in the folder \texttt{TaxFuncIntegr/Python} of the repository for this paper.}  Figure \ref{FigMicroTaxEst3D} gives a sense of how well these estimated functions are able to fit the data for our baseline policy.  Results are similar for the reform analyzed, as can be seen from the sum of squared errors from the non-linear least squares estimates presented in Table \ref{tab:tax_params}.

%   \begin{table}[htbp] \centering \captionsetup{width=5.2in}
%   \caption{\label{tab:tax_params}\textbf{Estimated baseline current law and policy change tax rate function $\tau_{s,t}(x,y)$ parameters for $s=42$, $t=2017$}}
%     \begin{threeparttable}
%     \begin{tabular}{>{\footnotesize}l |>{\footnotesize}r >{\footnotesize}r >{\footnotesize}r |>{\footnotesize}r >{\footnotesize}r >{\footnotesize}r}
%     \hline\hline
%     & \multicolumn{3}{c}{Baseline} & \multicolumn{3}{c}{Policy} \\
%     Parameter & \multicolumn{1}{c}{\footnotesize{ETR}} & \multicolumn{1}{c}{\footnotesize{MTRx}} & \multicolumn{1}{c}{\footnotesize{MTRy}} & \multicolumn{1}{c}{\footnotesize{ETR}} & \multicolumn{1}{c}{\footnotesize{MTRx}} & \multicolumn{1}{c}{\footnotesize{MTRy}} \\
%         \hline
%     $A$ & 6.28E-12 & 3.43E-23 & 4.32E-11 & 5.90E-12 & 3.43E-23 & 9.49E-11 \\
%     $B$ & 4.36E-05 & 4.50E-04 & 5.52E-05 & 3.61E-05 & 4.50E-04 & 2.48E-05 \\
%     $C$ & 1.04E-23 & 9.81E-12 & 5.62E-12 & 1.04E-23 & 1.13E-11 & 2.05E-11 \\
%     $D$ & 7.77E-09 & 5.30E-08 & 3.09E-06 & 9.14E-09 & 1.89E-06 & 7.08E-06 \\
%     $max\_{x}$ & 0.80  & 0.71  & 0.44  & 0.80  & 0.68  & 0.80 \\
%     $min\_{x}$ & -0.14 & -0.17 & 0.00E+00 & -0.14 & -0.17 & 0.00E+00 \\
%     $max\_{y}$ & 0.80  & 0.80  & 0.13  & 0.80  & 0.80  & 3.83E-03 \\
%     $min\_{y}$ & -0.15 & -0.42 & 0.00E+00 & -0.15 & -0.42 & 0.00E+00 \\
%     $shift\_{x}$ & 0.15  & 0.18  & 4.45E-03 & 0.15  & 0.18  & 0.01 \\
%     $shift\_{y}$ & 0.16  & 0.43  & 1.34E-03 & 0.16  & 0.43  & 3.83E-05 \\
%     $shift$ & -0.15 & -0.42 & 0.00E+00 & -0.15 & -0.42 & 0.00E+00 \\
%     $share$ & 0.84  & 0.96  & 0.86  & 0.83  & 0.96  & 0.85 \\
%     \hline
%     Obs. (N) & 3,105 & 3,105 & 1,990 & 3,105 & 3,105 & 1,990 \\
%     SSE   & 9122.68 & 15041.35 & 7756.54 & 8331.02 & 13921.42 & 7326.73 \\
%     \hline\hline
%     \end{tabular}
%     % \begin{tablenotes}
%     %   \scriptsize{\item[]Note: Maybe put sources here.}
%     % \end{tablenotes}
%     \end{threeparttable}
%   \end{table}


%   \subsection{Static revenue effects}

%     Using the web application for the \texttt{Tax-Calculator} at \href{www.ospc.org/TaxBrain}{www.ospc.org/TaxBrain}, we calculate the revenue effects of the proposal to reduce statutory marginal tax rates by 10 percent for each tax bracket and double the standard deduction for each filer type. Figure \ref{FigTaxBrainScore} shows a screen shot of the TaxBrain results page, which shows that the estimated static revenue change is about -\$2.4 trillion over 10 years (2017-2026).

%     \begin{figure}[htbp]\centering \captionsetup{width=6.0in}
%       \caption{\label{FigTaxBrainScore}\textbf{Screen shot from \texttt{Tax-Calculator} of static revenue estimate tax experiment}}
%       \fbox{\resizebox{6.0in}{3.5in}{\includegraphics{./images/TB_revenue_screenshot.png}}}
%     \end{figure}

%     We use the \texttt{Tax-Calculator} to find the new effective and marginal tax rates faced by each of the filers represented in the PUF-CPS match under the reform.  We use these data to fit tax functions over income by year and primary filer age. As explained above, the fitted functions are then used in our DGE model to analyze the effect of the reform on macroeconomic aggregates and prices.


%   \subsection{Macroeconomic effects}

%     Table \ref{tab:macro} displays the percentage changes in aggregate quantities and prices over the budget window and in the steady state.  These are computed through the DGE model described above, using the estimated tax functions for the baseline and policy tax parameters.  The parameters of the estimated tax functions are given in Table \ref{tab:tax_params}.

%     \begin{table}[htbp] \centering \captionsetup{width=5.8in}
%     \caption{\label{tab:macro}\textbf{Percent change in macroeconomic variables over the budget window and in steady-state from policy change}}
%       \begin{threeparttable}
%       \begin{tabular}{>{\scriptsize}l |>{\scriptsize}r >{\scriptsize}r >{\scriptsize}r >{\scriptsize}r >{\scriptsize}r >{\scriptsize}r >{\scriptsize}r >{\scriptsize}r >{\scriptsize}r >{\scriptsize}r |>{\scriptsize}r >{\scriptsize}r}
%       \hline\hline
%       \multicolumn{1}{c}{\scriptsize{Macroeconomic}} & & & & & & & & & & & \multicolumn{1}{c}{\scriptsize{2017-}} & \\[-2mm]
%       \multicolumn{1}{c}{\scriptsize{variables}}  & 2017  & 2018  & 2019  & 2020  & 2021  & 2022  & 2023  & 2024  & 2025 & 2026 & \multicolumn{1}{c}{\scriptsize{2026}} & \multicolumn{1}{c}{\scriptsize{SS}} \\
%         \hline
%     GDP   & -0.11 & 0.71  & 0.71  & 0.72  & 0.82  & 0.83  & 0.91  & 0.91  & 0.86  & 0.94  & 0.73  & 1.44 \\
%     Consumption & 0.44  & 0.47  & 0.51  & 0.57  & 0.61  & 0.65  & 0.69  & 0.72  & 0.77  & 0.79  & 0.62  & 1.16 \\
%     Investment & -1.36 & 1.24  & 1.16  & 1.06  & 1.30  & 1.22  & 1.38  & 1.34  & 1.08  & 1.27  & 0.98  & 2.09 \\
%     Hours Worked & -0.19 & 1.13  & 1.06  & 1.03  & 1.14  & 1.09  & 1.18  & 1.14  & 1.02  & 1.11  & 0.97  & 1.09 \\
%         \hline
%     Avg. Wage & 0.09  & -0.42 & -0.36 & -0.31 & -0.32 & -0.27 & -0.27 & -0.23 & -0.16 & -0.17 & -0.24 & 0.35 \\
%     Interest Rate & -0.30 & 1.45  & 1.23  & 1.07  & 1.12  & 0.95  & 0.96  & 0.81  & 0.57  & 0.63  & 0.85  & -1.33 \\
%         \hline
%             Total Taxes & -1.08 & -7.71 & -8.99 & -9.81 & -8.52 & -8.65 & -8.33 & -8.97 & -9.14 & -8.78 & -7.93 & -7.08 \\
%         \hline\hline
%       \end{tabular}
%       % \begin{tablenotes}
%       %   \scriptsize{\item[]Note: Maybe put sources here.}
%       % \end{tablenotes}
%       \end{threeparttable}
%     \end{table}


%   \subsection{Discussion}\label{SecDiscuss}

%     Qualitatively, the results of the macro model are consistent with economic theory. The reduction in marginal tax rates increases the incentives to work and save. We subsequently see increases aggregate hours worked and investment. As a result, both GDP and consumption increase. The wage and the interest rate tell an interesting story. Over the budget window, wages are falling and interest rates are increasing despite aggregate investment increasing by more than aggregate labor. This is a stock versus flow issue. The initial response to aggregate labor is bigger than that of the capital stock because it takes time for the capital from investment to accumulate. However, as the economy approaches the steady-state, the capital accumulation has caught up to the labor response resulting in a long run increase on the wage and decrease of interest rates.

%     Total tax revenues decrease due to the lower tax rates, but the increase in aggregate income that results from the additional investment and labor supply offset the revenue losses to some extent. Over the budget window, however, these effects are modest.  Both the static score from TaxBrain (shown in Figure \ref{FigTaxBrainScore}) show revenue losses of about eight percentage points from the reform.

%     A few caveats about the limitations of the model are in order. Of first order importance in determining the macroeconomic effects of changes in tax policy are the assumptions about how such tax changes are financed (see \citet{DM2011} for a comparison of results under different financing options). The DGE model used here has a simple balanced budget requirement for the government. This means that tax cuts, as we consider here, are financed by immediate reductions in the lump sum transfers the government makes to all households. The assumption is the most conducive to reductions in taxes providing positive macroeconomic effects. If these tax cuts were temporary and financed by future tax increases, the stimulative effects of such cuts would be substantially reduced. In fact, our long run increase in GDP of 1.4 percent lies well above the upper bounds found by \citet{DM2011} and \citet{CBO2004}. Such differences are largely driven by assumptions regarding how tax cuts are financed, but are also due to differences in model parameterizations and other features.

%     Not considered in this model, but also important in determining the macroeconomic effects of fiscal policy, are the policy responses of the central bank. Implicit in the results presented here is that the central bank does not respond to fiscal policy. If, for example, the central bank responded by holding interest rates constant, the supply side effects would be smaller and their would be less of a change in the macroeconomic aggregates.

%     Finally, one should note that while the levels of the macroeconomic aggregates change in the steady state as a result of tax policy, that long run growth rates do not. These long run growth rates are governed by exogenous changes in population growth and factor productivity. Thus these long run growth rates, in this framework, are not dependent on tax policy.




% \section{Conclusion}\label{sec:conclude}

%   Our goal was to introduce a methodology through which researchers and policy analysts could integrate the strengths of a microsimulation model of tax policy into aggregate models that allow one to understand the macroeconomic impacts of fiscal policy.  We apply this methodology by estimating the the revenue effects of a canonical example policy change in a microsimulation model and an overlapping generations DGE model. More broadly, we note that the methodology and underlying source code for the tax function estimation can be applied to link other microsimulation or macroeconomic models.

%   To the extent that a macroeconomic model has more degrees of heterogeneity than the one used as an example in this paper, one could add further dimensions to the tax function estimation. Consider a macroeconomic model with heterogeneity in households to account for differences betweens households with one and two earners. Assuming the microsimulation model could utilize data that allowed observation of this household characteristic, one could use the same methods proposed above to estimate tax functions separately for filing units with only a primary filer and those with primary and secondary filers. Thus these methods can be quite general and utilized by a wide class of models. The important consideration is the flexible specification of the tax functions that allow details of tax policy to be mapped to parametric functions used in macroeconomic models.

%   A compelling direction of future work integrating microsimulation models with general equilibrium models lies providing consistency between the macroeconomic assumptions underlying the microsimulation model and the macroeconomic effects found in the general equilibrium model. In our use of the \texttt{Tax-Calculator}, and in all static microsimulation models, it is assumed that macroeconomic variables either remain unchanged by policy experiments or a new path for the macroeconomic variables are produced by reduced form time series models. We could expand our methodology of integration of the microeconomic and macroeconomic models by providing for the following iterative procedure: Obtain solutions to the microsimulation model given an assumption about macroeconomic variables. Solve the macroeconomic model given the microeconomic results. Use the new macroeconomic variable time paths in the microsimulation model and re-compute the microsimulation results. Repeat until the macroeconomic assumptions of the microsimulation model are consistent with the macroeconomic results of the general equilibrium model.

%   \clearpage


\end{spacing}


\newpage
\bibliography{IncomeTrans}


% \newpage
% \renewcommand{\theequation}{A.\arabic{section}.\arabic{equation}}
%                                                  % redefine the command that creates the section number
% \renewcommand{\thesection}{A-\arabic{section}}   % redefine the command that creates the equation number
% \setcounter{equation}{0}                         % reset counter
% \setcounter{section}{0}                          % reset section number
% \section*{APPENDIX}                              % use *-form to suppress numbering

% \begin{spacing}{1.0}

% \section{Full Detail of Dynamic General Equilibrium Model}\label{AppDGE}

%   The dynamic general equilibrium (DGE) model in this paper is a close variant of the model used in \citet{DEMPRW2015}.\footnote{This paper can be downloaded at \href{https://sites.google.com/site/rickecon/WealthTax.pdf}{https://sites.google.com/site/rickecon/WealthTax.pdf}.} Our DGE model is comprised of heterogeneous households, perfectly competitive firms, and a government with a balanced budget requirement. A unit measure of identical firms make a static profit maximization decision in which they rent capital and hire labor to maximize profits given a Cobb-Douglas production function. The government levies taxes on households and makes lump sum transfers to households according to a balanced budget constraint. The model thus will present a relatively rich set of heterogeneity among households, but less heterogeneity in the production sector with only a trivial government sector. But the household sector is the most relevant to how we integrate these two models, even if other elements of the model such as government financing are important determinants of the macroeconomic outcomes.

%   Households are assumed to live for a maximum of $E+S$ periods. We define an age-$s$ household as being in youth and out of the workforce during ages $1\leq s\leq E$. We implement this dichotomy of being economically relevant by age in order to more easily match true population dynamics. households enter the workforce at age $E+1$ and remain in the workforce until they die or until the maximum age $E+S$. Because of mortality risk, households can leave both intentional bequests at the end of life ($s=E+S$) as well as accidental bequests if they die before the maximum age of $E+S$.

%   When households are born at age $s=1$, they are randomly assigned to one of $J$ lifetime earnings ability types. households remain in their assigned lifetime earnings ability group throughout their lives. Once born and assigned to a group, a household's lifetime earnings ability profile has a deterministic and known path. Related to hourly earnings, this process is calibrated to match the wage distribution by age in the United States.  Labor is endogenously supplied by households. Our calibration of the hourly earnings process allows for a skewed distribution of earnings that fits U.S. life-cycle hourly earnings data. The economic environment is one of incomplete markets because the overlapping generations structure prevents households from perfectly smoothing consumption.


%   \subsection{Population dynamics and lifetime earnings profiles}\label{SecPopDyn}

%     One of the contributions of this paper is to carefully calibrate and incorporate population dynamics into dynamic revenue estimation. \citet{NishiyamaSmetters:2007} note that including realistic population dynamics in large-scale overlapping generations models has been restricted to a small number of studies.\footnote{\citet{DeNardiEtAl:1999}, \citet{KotlikoffEtAl:2001}, and \citet{Nishiyama:2004} include the effect of nonstationary demographics on macroeconomic variables.} Two likely reasons are the following. First, population dynamics introduce an additional source of growth to the model that must be stationarized in order to compute equilibrium solutions. And population demographics take a large number of years to reach their steady-state. This significantly increases the computation time for transition path equilibrium solutions. A more detailed description of the population dynamics can be found in Appendix \ref{AppPopDyn}.

%     We define $\omega_{s,t}$ as the number of households of age $s$ alive at time $t$. A measure $\omega_{1,t}$ of households with heterogeneous working ability is born in each period $t$ and live for up to $E+S$ periods, with $S\geq 4$.\footnote{Theoretically, the model works without loss of generality for $S\geq 3$. However, because we are calibrating the ages outside of the economy to be one-fourth of $S$ (e.g., ages 21 to 100 in the economy, and ages 1 to 20 outside of the economy), we need $S$ to be at least 4.} Households are termed ``youth'', and do not participate in market activity during ages $1\leq s\leq E$. The households enter the workforce and economy in period $E+1$ and remain in the workforce until they unexpectedly die or live until age $s=E+S$. We model the population with households age $s\leq E$ outside of the workforce and economy in order most closely match the empirical population dynamics.

%     The population of agents of each age in each period $\omega_{s,t}$ evolves according to the following function,
%     \begin{equation}\label{EqPopLawofmotion}
%       \begin{split}
%         \omega_{1,t+1} &= (1 - \rho_0)\sum_{s=1}^{E+S} f_s\omega_{s,t} + i_1\omega_{1,t}\quad\forall t \\
%         \omega_{s+1,t+1} &= (1 - \rho_s)\omega_{s,t} + i_{s+1}\omega_{s+1,t}\quad\forall t\quad\text{and}\quad 1\leq s \leq E+S-1
%       \end{split}
%     \end{equation}
%     where $f_s\geq 0$ is an age-specific fertility rate, $i_s$ is an age-specific net immigration rate, $\rho_s$ is an age-specific mortality hazard rate, and $\rho_0$ is an infant mortality rate.\footnote{The parameter $\rho_s$ is the probability that a household of age $s$ dies before age $s+1$.} The total population in the economy $N_t$ at any period is simply the sum of households in the economy, the population growth rate in any period $t$ from the previous period $t-1$ is $g_{n,t}$, $\tilde{N}_t$ is the working age population, and $\tilde{g}_{n,t}$ is the working age population growth rate in any period $t$ from the previous period $t-1$.
%     \begin{equation}\label{EqPopN}
%       N_t\equiv\sum_{s=1}^{E+S} \omega_{s,t} \quad\forall t
%     \end{equation}
%     \begin{equation}\label{EqPopGrowth}
%       g_{n,t+1} \equiv \frac{N_{t+1}}{N_t} - 1 \quad\forall t
%     \end{equation}
%     \begin{equation}\label{EqPopNtil}
%       \tilde{N}_t\equiv\sum_{s=E+1}^{E+S} \omega_{s,t} \quad\forall t
%     \end{equation}
%     \begin{equation}\label{EqPopGrowthTil}
%       \tilde{g}_{n,t+1} \equiv \frac{\tilde{N}_{t+1}}{\tilde{N}_t} - 1 \quad\forall t
%     \end{equation}

%     At birth, a fraction $\lambda_j$ of the $\omega_{1,t}$ measure of new agents is randomly assigned to each of the $J$ lifetime income groups, indexed by $j=1,2,...J$, such that $\sum_{j=1}^J\lambda_j=1$. Note that lifetime income is endogenous in the model, therefore we define lifetime income groups by a particular path of earnings abilities. For each lifetime income group, the measure $\lambda_j\omega_{s,t}$ of households' effective labor units (which we also call ability) evolve deterministically according to $e_{j,s}$. This allows for heterogenous life cycle profiles of earnings ability across lifetime income groups over household working ages $E+1\leq s \leq E+S$. The exogenous earnings process $e_{j,s}$ is taken from \citet{DEMPRW2015}. The processes for the evolution of the population weights $\omega_{s,t}$ as well as lifetime earnings ability $e_{j,s}$ are exogenous inputs to the model.

%     \begin{figure}[htb]\centering \captionsetup{width=4.0in}
%       \caption{\label{FigLogAbility}\textbf{Exogenous life cycle income ability paths $\log(e_{j,s})$ with $S=80$ and $J=7$}}
%       \fbox{\resizebox{4.0in}{2.7in}{\includegraphics{./images/ability_log_2D.png}}}
%     \end{figure}

%     Figure \ref{FigLogAbility} shows the calibrated trajectory of effective labor units (ability) $e_{j,s}\in\mathcal{E}\subset\mathbb{R}_{++}$ by age $s$ for each type $j$ for lifetime income distribution $\{\lambda_j\}_{j=1}^7 = [0.25,0.25,0.20,0.10,0.10,0.09,0.01]$. We show effective labor units in logarithms because the difference in levels between the top one percent and the rest of the distribution is so large. All model households have the same time endowment and receive the same wage per effective labor unit, but some are endowed with more effective labor units. We utilize a measure of lifetime income, by using potential lifetime earnings, that allows us to define income groups in a way that accounts for the fact that earnings of households observed in the data are endogenous.  It is in this way that we are able to calibrate the exogenous lifetime earnings profiles from the model with their data counterparts.


%   \subsection{The household's problem}\label{SecIndProb}

%     Households are endowed with a measure of time, denoted by $\tilde{l}$, in each period $t$.  Households choose how much of that time to allocate between labor $n_{j,s,t}$ and leisure $l_{j,s,t}$ in each period. That is, a household's labor and leisure choice is constrained by its total time endowment.  The total time endowment is constraint is identical across all households.
%     \begin{equation}\label{EqLabConstr}
%       n_{j,s,t} + l_{j,s,t} = \tilde{l}
%     \end{equation}
%     At time $t$, all age-$s$ households with ability $e_{j,s}$ know the real wage rate, $w_t$, and know the one-period real net interest rate, $r_t$, on bond holdings, $b_{j,s,t}$, that mature at the beginning of period $t$. They also receive accidental and intentional bequests. They choose how much to consume, $c_{j,s,t}$, how much to save for the next period by loaning capital to firms in the form of a one-period bond, $b_{j,s+1,t+1}$, and how much to work, $n_{j,s,t}$, in order to maximize expected lifetime utility of the following form,
%     \begin{equation}\label{EqUtilMax}
%       \begin{split}
%         &U_{j,s,t} = \sum_{u=0}^{E+S-s}\beta^u\left[\prod_{v=s}^{s+u-1}(1-\rho_v)\right] u\left(c_{j,s+u,t+u},n_{j,s+u,t+u},b_{j,s+u+1,t+u+1}\right) \\
%         &\text{and} \quad u\left(c_{j,s,t},n_{j,s,t},b_{j,s+1,t+1}\right) = \frac{\left(c_{j,s,t}\right)^{1-\sigma} - 1}{1-\sigma} ... \\
%         &\qquad\qquad + e^{g_y t(1-\sigma)}\chi^n_s\left(b\left[1 - \left(\frac{n_{j,s,t}}{\tilde{l}}\right)^\upsilon\right]^\frac{1}{\upsilon}\right) + \rho_s\chi^b_j\frac{\left(b_{j,s+1,t+1}\right)^{1-\sigma} - 1}{1-\sigma} \\
%         &\quad\quad\quad\quad\quad\quad\quad\quad\quad\quad\quad\quad\quad\quad\quad\quad\quad\quad\quad\forall j,t\quad\text{and}\:E+1\leq s\leq E+S
%       \end{split}
%     \end{equation}
%     where $\sigma\geq 1$ is the coefficient of relative risk aversion on consumption and on utility from bequests, $\beta\in(0,1)$ is the agent's discount factor, and the product term in brackets depreciates the household's discount factor by the cumulative mortality rate.

%     The disutility of labor term in the period utility function looks nonstandard, but is simply the upper right quadrant of an ellipse. \citet{EvansPhillips:2016} show that this functional form closely approximates the standard constant relative risk aversion (CRRA) utility of leisure and constant Frisch elasticity (CFE) disutility of labor functional forms. It also provides Inada conditions for the upper and lower bounds of labor supply, which make the computation of the solution more tractable. We estimate the parameters of the elliptical disutility of labor function in the second term of the period utility function in \eqref{EqUtilMax} to be $[b,\upsilon]=[0.573, 2.856]$, which closely matches a CFE functional form with a Frisch elasticity of 0.41.

%     The term $\chi^n_s$ is a constant term that varies by age $s$ influencing the disutility of labor relative to the other arguments in the period utility function,\footnote{\citet{DEMPRW2015} calibrate $\chi^n_s$ and $\chi^b_j$ to match average labor hours by age and some moments of the distribution of wealth.} and $g_y$ is a constant growth rate of labor augmenting technological progress, which we explain in Section \ref{SecFirms}.\footnote{The term with the growth rate $e^{g_y t(1-\sigma)}$ must be included in the period utility function because consumption and bequests will be growing at rate $g_y$ and this term stationarizes the household Euler equation by making the marginal disutility of labor grow at the same rate as the marginal benefits of consumption and bequests.  This is the same balanced growth technique as that used in \citet{MertensRavn:2011}.}

%     The last term in \eqref{EqUtilMax} incorporates a warm-glow bequest motive in which households value having savings to bequeath to the next generation in the chance they die before the next period. Including this term is essential to generating the positive wealth levels across the life cycle and across abilities that exist in the data. In addition, the term $\chi^b_j$ is a constant term that varies by lifetime income group $j$ influencing the marginal utility of bequests, $b_{j,s+1,t+1}$, relative to the other arguments in the period utility function. Allowing the $\chi^b_j$ scale parameter on the warm glow bequest motive vary by lifetime income group is critical for matching the distribution of wealth. As was mentioned in Section \ref{SecPopDyn}, households in the model have no income uncertainty because each lifetime earnings path $e_{j,s}$ deterministic, model agents thus hold no precautionary savings. Calibrating the $\chi^b_j$ for each income group $j$ captures in a reduced form way some of the characteristics that household income risk provides.

%     The parameter $\sigma\geq 1$ is the coefficient of relative risk aversion on bequests, and the mortality rate $\rho_s$ appropriately discounts the value of this term.\footnote{It is necessary for the coefficient of relative risk aversion $\sigma$ to be the same on both the utility of consumption and the utility of bequests. If not, the resulting Euler equations are not stationarizable.} Note that, because of this bequest motive, households in the last period of their lives ($s=S$) will die with positive savings $b>0$. Also note that the CRRA utility of bequests term prohibits negative wealth holdings in the model. However, we do not find this to be a strong restriction since, when one aggregates data on wealth by age and income percentile,  only the young in the lowest quartile show negative wealth holdings.

%     The per-period budget constraints for each agent are given by the following equation,
%     \begin{equation}\label{EqBC}
%       \begin{split}
%         &c_{j,s,t} + b_{j,s+1,t+1} \leq \left(1 + r_t\right) b_{j,s,t} + w_t e_{j,s}n_{j,s,t} + \frac{BQ_{j,t}}{\lambda_j\tilde{N}_t} - T_{s,t} \\
%         &\qquad\qquad\qquad\text{where}\quad b_{j,E+1,t} = 0\quad\text{for} \quad E+1\leq s \leq E+S \quad \forall j,t
%       \end{split}
%     \end{equation}
%     where $\tilde{N}_t$ is the total working age population at time $t$ defined in \eqref{EqPopNtil} and $\lambda_j\tilde{N}_t$ is the number of the total working households of type $j$ in period $t$. The price of consumption is normalized to one, so $w_t$ is the real wage and $r_t$ is the net real interest rate. The term $BQ_{j,t}$ represents total bequests from households in income group $j$ who died at the end of period $t-1$. $T_{s,t}$ is a function representing net taxes paid, which we specify more fully in equation \eqref{EqNetTaxLiab}.

%     Implicit in the period budget constraint \eqref{EqBC} is a strong assumption about the distribution of bequests. We assume that bequests are distributed evenly across all ages to those in the same lifetime income group. It is difficult to precisely calibrate the distribution of bequests from the data, both across income types $j$ and across ages $s$. However, the assumptions about the bequest motive as well as how bequests are distributed are clearly important modeling decisions. Our current specification of bequests is the most persistent, which should make wealth inequality more persistent relative to other bequest specifications.\footnote{Another allocation rule at the opposite extreme would be to equally divide all bequests among all surviving households. An intermediate rule would be some kind of distribution of bequests with most going to ones own type and a declining proportion going to the other types.} A large number of papers study the effects of different bequest motives and specifications on the distribution of wealth, though there is no consensus regarding the true bequest transmission process.\footnote{See \citet{DeNardiYang:2014}, \citet{DeNardi:2004}, \citet{Nishiyama:2002}, \citet{Laitner:2001}, \citet{GokhaleEtAl:2000}, \citet{GaleScholz:1994}, \citet{Hurd:1989}, \citet{VentiWise:1988}, \citet{KotlikoffSummers:1981}, and \citet{Wolff:2015}.}

%     Because the form of the period utility function in \eqref{EqUtilMax} ensures that $b_{j,s,t}>0$ for all $j$, $s$, and $t$, total bequests will always be positive $BQ_{j,t}>0$ for all $j$ and $t$.
%     \begin{equation}\label{EqTotBeq}
%       BQ_{j,t+1} = (1+r_{t+1})\lambda_j\left(\sum_{s=E+1}^{E+S}\rho_s\omega_{s,t}b_{j,s+1,t+1}\right) \quad\forall j,t
%     \end{equation}
%     In addition to each the budget constraint in each period, the utility function \eqref{EqUtilMax} imposes nonnegative consumption through infinite marginal utility, and the elliptical utility of leisure ensures household labor and leisure must be strictly nonnegative $n_{j,s,t},l_{j,s,t}> 0$. Because household savings or wealth is always strictly positive, the aggregate capital stock is always positive.\footnote{An alternative would be to allow for household borrowing as long as an aggregate capital constraint $K_{t}>0$ for all $t$ is satisfied.} An interior solution to the household's problem \eqref{EqUtilMax} is assured.


%     The solution to the lifetime maximization problem \eqref{EqUtilMax} of household with ability $j$ subject to the per-period budget constraint \eqref{EqBC} and the specification of taxes in \eqref{EqNetTaxLiab} is a system of $2S$ Euler equations. The $S$ static first order conditions for labor supply $n_{j,s,t}$ are the following,
%     \begin{equation}\label{EqEulerLabGen}
%       \begin{split}
%         &(c_{j,s,t})^{-\sigma}\Biggl(w_t e_{j,s} - \frac{\partial T_{s,t}}{\partial n_{j,s,t}}\Biggr) = e^{g_y t(1-\sigma)}\chi^n_{s}\biggl(\frac{b}{\tilde{l}}\biggr)\biggl(\frac{n_{j,s,t}}{\tilde{l}}\biggr)^{v-1}\Biggl[1 - \biggl(\frac{n_{j,s,t}}{\tilde{l}}\biggr)^\upsilon\Biggr]^{\frac{1-v}{v}} \\
%         &\qquad\qquad\qquad\qquad\qquad\qquad\qquad\qquad\qquad\forall j,t, \quad\text{and}\quad E+1\leq s\leq E+S \\
%         &\qquad\text{where}\quad c_{j,s,t} = \left(1 + r_t\right) b_{j,s,t} + w_t e_{j,s}n_{j,s,t} + \frac{BQ_{j,t}}{\lambda_j\tilde{N}_t} - b_{j,s+1,t+1} - T_{s,t} \\
%         &\qquad\text{and}\quad b_{j,E+1,t} = 0 \quad\forall j,t
%       \end{split}
%     \end{equation}
%     where the marginal tax rate with respect to labor supply $\frac{\partial T_{s,t}}{\partial n_{j,s,t}}$ is described in equation \eqref{EqMTR_lab}.\footnote{We also have to use a parameter $factor$ that multiplies the model labor income and the model capital income in the tax function in order to match their levels to the corresponding average levels in the microsimulation model data. This is described in more detail in Section \ref{SecIntegrMicrosim}.}

%     An household also has $S-1$ dynamic Euler equations that govern his saving decisions, $b_{j,s+1,t+1}$, with the included precautionary bequest saving in case of unexpected death. These are given by:
%     \begin{equation}\label{EqEulerSavGen}
%       \begin{split}
%         &(c_{j,s,t})^{-\sigma} = \rho_s\chi^b_j\bigl(b_{j,s+1,t+1}\bigr)^{-\sigma} + \beta(1-\rho_s)(c_{j,s+1,t+1})^{-\sigma}\Biggl[(1 + r_{t+1}) - \frac{\partial T_{s+1,t+1}}{\partial b_{j,s+1,t+1}}\Biggr] \\
%         &\qquad\qquad\qquad\qquad\qquad\qquad\qquad\qquad\forall j,t,\quad\text{and}\quad E+1\leq s \leq E+S-1
%       \end{split}
%     \end{equation}
%     where the marginal tax rate with respect to savings $\frac{\partial T_{s,t}}{\partial b_{j,s,t}}$ is described in equation \eqref{EqMTR_cap}. Lastly, each household also has one static first order condition for the last period of life $s=E+S$, which governs how much to bequeath to the following generation given that the household will die with certainty. This condition is simply equation \eqref{EqEulerSavGen} with $\rho_s=1$.
%     \begin{equation}\label{EqEulerSavEpS}
%       (c_{j,E+S,t})^{-\sigma} = \chi^b_j(b_{j,E+S+1,t+1})^{-\sigma} \quad\forall j,t
%     \end{equation}

%     Define $\bm{\hat{\Gamma}}_t$ as the distribution of stationary household savings across households at time $t$, including the intentional bequests of the oldest cohort.
%     \begin{equation}\label{EqSavDist}
%       \bm{\hat{\Gamma}}_t \equiv \Bigl\{\bigl\{\hat{b}_{j,s,t}\bigr\}_{j=1}^J\Bigr\}_{s=E+2}^{E+S+1} \quad\forall t
%     \end{equation}
%     As will be shown in Section \ref{SecMCEqlbm}, the state in every period $t$ for the entire equilibrium system described in the stationary, non-steady-state equilibrium characterized in Definition \ref{DefEquilNonSS} is the stationary distribution of household savings $\bm{\hat{\Gamma}}_t$ from \eqref{EqSavDist}. Because households must forecast wages, interest rates, and aggregate bequests received in every period in order to solve their optimal decisions and because each of those future variables depends on the entire distribution of savings in the future, we must assume some household beliefs about how the entire distribution will evolve over time. Let general beliefs about the future distribution of capital in period $t+u$ be characterized by the operator $\Omega(\cdot)$ such that:
%     \begin{equation}\label{EqBeliefs}
%       \bm{\hat{\Gamma}^e_{t+u}} = \Omega^u\left(\bm{\hat{\Gamma}_t}\right) \quad \forall t, \quad u\geq 1
%     \end{equation}
%     where the $e$ superscript signifies that $\hat{\Gamma}^e_{t+u}$ is the expected distribution of wealth at time $t+u$ based on general beliefs $\Omega(\cdot)$ that are not constrained to be correct.\footnote{In Section \ref{SecMCEqlbm} we will assume that beliefs are correct (rational expectations) for the stationary non-steady-state equilibrium in Definition \ref{DefEquilNonSS}.}


%   \subsection{Firm problem}\label{SecFirms}

%     A unit measure of identical, perfectly competitive firms exist in the economy. The representative firm is characterized by the following Cobb-Douglas production technology,
%     \begin{equation}\label{EqCobbDougProd}
%        Y_t = Z K_t^\alpha\left(e^{g_y t}L_t\right)^{1-\alpha} \quad \forall t
%     \end{equation}
%     where $Z$ is the measure of total factor productivity, $\alpha\in(0,1)$ is the capital share of income, $g_y$ is the constant growth rate of labor augmenting technological change, and $L_t$ is aggregate labor measured in efficiency units. The firm uses this technology to produce a homogeneous output which is consumed by households and used in firm investment.  The interest rate $r_t$ paid to the owners of capital is the real interest rate net of depreciation. The real wage is $w_t$.  The real profit function of the firm is the following.
%     \begin{equation}\label{EqFirmProfit}
%        \text{Real Profits} = Z K_t^\alpha\left(e^{g_y t}L_t\right)^{1-\alpha} - (r_t + \delta)K_t - w_t L_t
%     \end{equation}
%     As in the household budget constraint \eqref{EqBC}, note that the price output has been normalized to one.

%     Profit maximization results in the real wage, $w_t$, and the real rental rate of capital $r_t$ being determined by the marginal products of labor and capital, respectively:
%     \begin{align}
%        w_t &= (1-\alpha)\frac{Y_t}{L_t} \quad \forall t \label{EqFOCwage}\\
%        r_t &= \alpha\frac{Y_t}{K_t} - \delta \quad\:\:\: \forall t \label{EqFOCrate}
%     \end{align}


%   \subsection{Government fiscal policy}\label{SecGovt}

%     The government is represented by a balanced budget constraint. The government collects taxes $\tau_{s,t}(x,y)(x+y)$ from all households and divides total revenues equally among all economically active households in the economy to determine the lump-sum transfer.
%     \begin{equation}\label{EqGovtBC}
%       T^H_t = \frac{1}{\tilde N_t} \sum_s \sum_j \omega_{s,t}\lambda_j\tau_{s,t}(w_t e_{j,s}n_{j,s,t}, r_t b_{j,s,t})\bigl(w_t e_{j,s}n_{j,s,t} + r_t b_{j,s,t}\bigr)
%     \end{equation}

%     Lump sum transfers have an impact on the distribution of income and wealth. Note that if one constrains policy experiments to have the same steady-state revenue impact, the changes in inequality in economic outcomes due to changes in government transfers is equivalent in each policy experiment in the steady-state.


%   \subsection{Market clearing and stationary equilibrium}\label{SecMCEqlbm}

%     Labor market clearing requires that aggregate labor demand $L_t$ measured in efficiency units equal the sum of household efficiency labor supplied $e_{j,s}n_{j,s,t}$. Capital market clearing requires that aggregate capital demand $K_t$ equal the sum of capital investment by households $b_{j,s,t}$. Aggregate consumption $C_t$ is defined as the sum of all household consumptions, and aggregate investment is defined by the resource constraint $Y_t = C_t + I_t$ as shown in \eqref{EqMktClrGoods}.
%     \begin{align}
%       L_t &= \sum_{s=E+1}^{E+S}\sum_{j=1}^{J} \omega_{s,t}\lambda_j e_{j,s}n_{j,s,t} \quad \forall t \label{EqMktClrLab} \\
%       K_t &= \sum_{s=E+2}^{E+S+1}\sum_{j=1}^{J}\Bigl(\omega_{s-1,t-1}\lambda_j b_{j,s,t} + i_s\omega_{s,t-1}\lambda_j b_{j,s,t}\Bigr) \quad \forall t \label{EqMktClrCap} \\
%       \begin{split}
%         Y_t &= C_t + K_{t+1} - \biggl(\sum_{s=E+2}^{E+S+1}\sum_{j=1}^{J}i_s\omega_{s,t}\lambda_j b_{j,s,t+1}\biggr) - (1-\delta)K_t \quad\forall t \\
%         &\quad\text{where}\quad C_t \equiv \sum_{s=E+1}^{E+S}\sum_{j=1}^{J}\omega_{s,t}\lambda_j c_{j,s,t}
%       \end{split} \label{EqMktClrGoods}
%     \end{align}
%     Note that the extra terms with the immigration rate $i_s$ in the capital market clearing equation \eqref{EqMktClrCap} and the resource constraint \eqref{EqMktClrGoods} accounts for the assumption that age-$s$ immigrants in period $t$ bring with them (or take with them in the case of out-migration) the same amount of capital as their domestic counterparts of the same age.

%     The usual definition of equilibrium would be allocations and prices such that households optimize \eqref{EqEulerLabGen}, \eqref{EqEulerSavGen}, and \eqref{EqEulerSavEpS}, firms optimize \eqref{EqFOCwage} and \eqref{EqFOCrate}, and markets clear \eqref{EqMktClrLab} and \eqref{EqMktClrCap}. However, the variables in the equations characterizing the equilibrium are potentially non-stationary due to the growth rate in the total population $g_{n,t}$ each period coming from the cohort growth rates in \eqref{EqPopLawofmotion} and from the deterministic growth rate of labor augmenting technological change $g_y$ in \eqref{EqCobbDougProd}.

%     \begin{table}[htbp] \centering \captionsetup{width=3.3in}
%     \caption{\label{TabStatVars}\textbf{Stationary variable definitions}}
%       \begin{threeparttable}
%       \begin{tabular}{>{\small}c >{\small}c >{\small}c |>{\small}c}
%         \hline\hline
%         \multicolumn{3}{c}{Sources of growth} & Not \\
%         & & & \\[-4mm]
%         $e^{g_y t}$ & $\tilde{N}_t$ & $e^{g_y t}\tilde{N}_t$ & growing\tnote{a} \\
%         \hline
%         & & \\[-4mm]
%         $\hat{c}_{j,s,t}\equiv\frac{c_{j,s,t}}{e^{g_y t}}$ & $\hat{\omega}_{s,t}\equiv\frac{\omega_{s,t}}{\tilde{N}_t}$ & $\hat{Y}_t\equiv\frac{Y_t}{e^{g_y t}\tilde{N}_t}$ & $n_{j,s,t}$ \\[2mm]
%         $\hat{b}_{j,s,t}\equiv\frac{b_{j,s,t}}{e^{g_y t}}$ & $\hat{L}_t\equiv\frac{L_t}{\tilde{N}_t}$ & $\hat{K}_t\equiv\frac{K_t}{e^{g_y t}\tilde{N}_t}$ & $r_t$ \\[2mm]
%         $\hat{w}_t\equiv\frac{w_t}{e^{g_y t}}$ &  & $\hat{BQ}_{j,t}\equiv\frac{BQ_{j,t}}{e^{g_y t}\tilde{N}_t}$ &  \\[2mm]
%         $\hat{y}_{j,s,t}\equiv\frac{y_{j,s,t}}{e^{g_y t}}$ &  & $\hat{C}_t\equiv \frac{C_t}{e^{g_y t}\tilde{N}_t}$  &  \\[2mm]
%         $\hat{T}_{s,t}\equiv\frac{T_{j,s,t}}{e^{g_y t}}$ &  &  &  \\[2mm]
%         \hline\hline
%       \end{tabular}
%       \begin{tablenotes}
%         \scriptsize{\item[a]The interest rate $r_t$ in \eqref{EqFOCrate} is already stationary because $Y_t$ and $K_t$ grow at the same rate. household labor supply $n_{j,s,t}$ is stationary.}
%       \end{tablenotes}
%       \end{threeparttable}
%     \end{table}

%     Table \ref{TabStatVars} characterizes the stationary versions of the variables of the model in terms of the variables that grow because of labor augmenting technological change, population growth, both, or none. With the definitions in Table \ref{TabStatVars}, it can be shown that the equations  characterizing the equilibrium can be written in stationary form in the following way. The static and intertemporal first-order conditions from the household's optimization problem corresponding to \eqref{EqEulerLabGen}, \eqref{EqEulerSavGen}, and \eqref{EqEulerSavEpS} are the following:

%     \begin{equation}\label{EqEulerLabStat}
%       \begin{split}
%         &(\hat{c}_{j,s,t})^{-\sigma}\Biggl(\hat{w}_t e_{j,s} - \frac{\partial\hat{T}_{s,t}}{\partial n_{j,s,t}}\Biggr) = \chi^n_{s}\biggl(\frac{b}{\tilde{l}}\biggr)\biggl(\frac{n_{j,s,t}}{\tilde{l}}\biggr)^{\upsilon-1}\Biggl[1 - \biggl(\frac{n_{j,s,t}}{\tilde{l}}\biggr)^\upsilon\Biggr]^{\frac{1-\upsilon}{\upsilon}} \\
%         &\qquad\qquad\qquad\qquad\qquad\qquad\qquad\qquad\forall j,t, \quad\text{and}\quad E+1\leq s\leq E+S \\
%         &\qquad\text{where}\quad \hat{c}_{j,s,t} = \left(1 + r_t\right)\hat{b}_{j,s,t} + \hat{w}_t e_{j,s}n_{j,s,t} + \frac{\hat{BQ}_{j,t}}{\lambda_j} - e^{g_y}\hat{b}_{j,s+1,t+1} - \hat{T}_{s,t} \\
%         &\qquad\text{and}\quad \hat{b}_{j,E+1,t} = 0 \quad\forall j,t
%       \end{split}
%     \end{equation}

%     \begin{equation}\label{EqEulerSavStat}
%       \begin{split}
%         &(\hat{c}_{j,s,t})^{-\sigma} = ... \\
%         &e^{-g_y\sigma}\Biggl(\rho_s\chi^b_j \bigl(\hat{b}_{j,s+1,t+1}\bigr)^{-\sigma} + \beta(1-\rho_s)(\hat{c}_{j,s+1,t+1})^{-\sigma}\Biggl[1 + r_{t+1} - \frac{\partial \hat{T}_{s+1,t+1}}{\partial \hat{b}_{j,s+1,t+1}}\Biggr]\Biggr) \\
%         &\qquad\qquad\qquad\qquad\qquad\qquad\qquad\qquad\forall j,t,\quad\text{and}\quad E+1\leq s \leq E+S-1
%       \end{split}
%     \end{equation}

%     \begin{equation}\label{EqEulerSavEpSstat}
%       (\hat{c}_{j,E+S,t})^{-\sigma} = \chi^b_j e^{-g_y\sigma}(\hat{b}_{j,E+S+1,t+1})^{-\sigma} \quad\forall j,t
%     \end{equation}

%     The stationary firm first order conditions for optimal labor and capital demand corresponding to \eqref{EqFOCwage} and \eqref{EqFOCrate} are the following.
%     \begin{equation}\label{EqFOCwageStat}
%        \hat{w}_t = (1-\alpha)\frac{\hat{Y}_t}{\hat{L}_t} \quad \forall t
%     \end{equation}
%     \begin{equation}\tag{\ref{EqFOCrate}}
%        r_t = \alpha\frac{\hat{Y}_t}{\hat{K}_t} - \delta = \alpha\frac{Y_t}{K_t} - \delta \quad \forall t
%     \end{equation}
%     And the two stationary market clearing conditions corresponding to \eqref{EqMktClrLab} and \eqref{EqMktClrCap}---with the goods market clearing by Walras' Law---are the following.
%     \begin{align}
%       \hat{L}_t &= \sum_{s=E+1}^{E+S}\sum_{j=1}^{J} \hat{\omega}_{s,t}\lambda_j e_{j,s}n_{j,s,t} \quad \forall t \label{EqMktClrLabStat} \\
%       \hat{K}_t &= \frac{1}{1 + \tilde{g}_{n,t}}\left(\sum_{s=E+2}^{E+S+1}\sum_{j=1}^{J}\hat{\omega}_{s-1,t-1}\lambda_j \hat{b}_{j,s,t} - i_s\hat{\omega}_{s,t-1}\lambda_j\hat{b}_{j,s,t}\right) \quad \forall t \label{EqMktClrCapStat}
%     \end{align}
%     where $\tilde{g}_{n,t}$ is the growth rate in the working age population between periods $t-1$ and $t$ described in \eqref{EqPopGrowthTil}. The stationary version of the goods market clearing condition (aggregate resource constraint) is the following.
%     \begin{equation}\label{EqMktClrGoodsStat}
%       \hat{Y}_t = \hat{C}_t + e^{g_y}(1 + \tilde{g}_{n,t+1})\hat{K}_{t+1} - e^{g_y}\left(\sum_{s=E+2}^{E+S+1}\sum_{j=1}^{J}i_s\hat{\omega}_{s,t}\lambda_j\hat{b}_{j,s,t+1}\right) - (1-\delta)\hat{K}_t \quad \forall t
%     \end{equation}
%     It is also important to note the stationary version of the characterization of total bequests $BQ_{j,t+1}$ from \eqref{EqTotBeq} and for the government budget constraint in \eqref{EqGovtBC}.
%     \begin{equation}\label{EqTotBeqStat}
%       \hat{BQ}_{j,t+1} = \frac{(1+r_{t+1})\lambda_j}{1+\tilde{g}_{n,t+1}}\left(\sum_{s=E+1}^{E+S}\rho_s\hat{\omega}_{s,t}\hat{b}_{j,s+1,t+1}\right) \quad\forall j,t
%     \end{equation}
%     \begin{equation}\label{EqGovtBCstat}
%       \hat{T}^H_t = \sum_s \sum_j \hat{\omega}_{s,t}\lambda_j \hat{T}_{s,t}
%     \end{equation}

%     We can now define the stationary steady-state equilibrium for this economy in the following way.

%     \vspace{7mm}
%     \end{spacing}
%     \hrule
%     \begin{definition}[\textbf{Stationary steady-state equilibrium}]\label{DefEquilSS}
%       A non-autarkic stationary steady-state equilibrium in the overlapping generations model with $S$-period lived agents and heterogeneous ability $e_{j,s}$ is defined as constant allocations $n_{j,s,t}=\bar{n}_{j,s}$ and $\hat{b}_{j,s+1,t+1}=\bar{b}_{j,s+1}$ and constant prices $\hat{w}_t=\bar{w}$ and $r_t=\bar{r}$ for all $j$, $s$, and $t$ such that the following conditions hold:
%        \begin{enumerate}
%           \item Households optimize according to \eqref{EqEulerLabStat}, \eqref{EqEulerSavStat}, and \eqref{EqEulerSavEpSstat},
%           \item Firms optimize according to \eqref{EqFOCwageStat} and \eqref{EqFOCrate},
%           \item Markets clear according to \eqref{EqMktClrLabStat} and \eqref{EqMktClrCapStat}, and
%           \item The population has reached its stationary steady state distribution $\bar{\omega}_s$ for all ages $s$, characterized in Appendix \ref{AppPopDyn}.
%        \end{enumerate}
%     \end{definition}
%     \hrule
%     \begin{spacing}{1.5}
%     \vspace{10mm}

%     The steady-state equilibrium is characterized by the system of $2JS$ equations and $2JS$ unknowns $\bar{n}_{j,s}$ and $\bar{b}_{j,s+1}$. Appendix \ref{AppSSsolve} details how to solve for the steady-state equilibrium.

%     The non-steady state equilibrium is characterized by $2JST$ equations and $2JST$ unknowns, where $T$ is the number of periods along the transition path from the current state to the steady state.  The definition of the stationary non-steady-state equilibrium is similar to Definition \ref{DefEquilSS}, with the stationary steady-state equilibrium definition being a special case of the stationary non-steady-state equilibrium.

%     \vspace{7mm}

%     \hrule
%     \begin{definition}[\textbf{Stationary non-steady-state equilibrium}]\label{DefEquilNonSS}
%       A non-autarkic stationary non-steady-state equilibrium in the overlapping generations model with $S$-period lived agents and heterogeneous ability $e_{j,s}$ is defined as allocations $n_{j,s,t}$ and $\hat{b}_{j,s+1,t+1}$ and prices $\hat{w}_t$ and $r_t$ for all $j$, $s$, and $t$ such that the following conditions hold:
%        \begin{enumerate}
%           \item Households have symmetric beliefs $\Omega(\cdot)$ about the evolution of the distribution of savings, and those beliefs about the future distribution of savings equal the realized outcome (rational expectations),
%             \begin{equation*}
%               \bm{\hat{\Gamma}}_{t+u} = \bm{\hat{\Gamma}}^e_{t+u} = \Omega^u\left( \bm{\hat{\Gamma}}_t\right) \quad\forall t, \quad u\geq 1
%             \end{equation*}
%           \item Households optimize according to \eqref{EqEulerLabStat}, \eqref{EqEulerSavStat}, and \eqref{EqEulerSavEpSstat}
%           \item Firms optimize according to \eqref{EqFOCwageStat} and \eqref{EqFOCrate}, and
%           \item Markets clear according to \eqref{EqMktClrLabStat} and \eqref{EqMktClrCapStat}.
%        \end{enumerate}
%     \end{definition}
%     \hrule

%     \vspace{10mm}

%     We describe the methodology to compute the solution to the non-steady-state equilibrium to Appendix \ref{AppNonSSsolve}.  We use the equilibrium transition path solution to find effects of tax policies on macroeconomic variables over the budget window.


%   \subsection{Calibration}\label{SecCalib}

%     Table \ref{TabExogVars} shows the calibrated values for the exogenous variables and parameters taken from \citet{DEMPRW2015}.

%     \begin{table}[htbp] \centering \captionsetup{width=4.7in}
%     \caption{\label{TabExogVars}\textbf{List of exogenous variables and baseline calibration values}}
%       \begin{threeparttable}
%       \begin{tabular}{>{\footnotesize}c |>{\footnotesize}l |>{\footnotesize}c}
%         \hline\hline
%         Symbol & \quad\quad\quad\quad Description & Value \\
%         \hline
%         $\bm{\hat{\Gamma}}_1$ & Initial distribution of savings & $\bm{\bar{\Gamma}}$ \\
%         $N_0$ & Initial population & 1 \\
%         $\{\omega_{s,0}\}_{s=1}^S$ & Initial population by age & (see App. \ref{AppPopDyn}) \\
%         $\{f_s\}_{s=1}^S$ & Fertility rates by age & (see App. \ref{AppPopDyn}) \\
%         $\{i_s\}_{s=1}^S$ & Immigration rates by age & (see App. \ref{AppPopDyn}) \\
%         $\{\rho_s\}_{s=1}^S$ & Mortality rates by age & (see App. \ref{AppPopDyn}) \\
%         $\{e_{j,s}\}_{j,s=1}^{J,S}$ & Deterministic ability process & (see \citealp{DEMPRW2015}) \\
%         $\{\lambda_j\}_{j=1}^J$ & Lifetime income group percentages & $[0.25,0.25,0.20,0.10,0.10,0.09,0.01]$ \\
%         $J$ & Number of lifetime income groups & 7 \\
%         $S$ & Maximum periods in economically active & 80 \\[-2mm]
%         &\quad household life & \\
%         $E$ & Number of periods of youth economically & $\text{round}\left(\frac{S}{4}\right)=20$ \\[-2mm]
%         & \quad outside the model & \\
%         $R$ & Retirement age (period) & $E+\text{round}\left(\frac{9}{16}S\right)=65$ \\
%         \hline
%         $\tilde{l}$ & Maximum hours of labor supply & 1 \\
%         $\beta$ & Discount factor & $(0.96)^\frac{80}{S}$ \\
%         $\sigma$ & Coefficient of constant relative risk aversion & 1.5 \\
%         $b$ & Scale parameter in utility of leisure & 0.573 \\
%         $\upsilon$ & Shape parameter in utility of leisure & 2.856 \\
%         $\chi^n_s$ & Disutility of labor level parameters & [19.041, 76.623] \\
%         $\chi^b_j$ & Utility of bequests level parameters &  $[9.264 \times 10^{-5}, 118,648]$ \\ %$1.0 \ \forall j$ \\
%         \hline
%         $Z$ & Level parameter in production function & 1.0 \\
%         $\alpha$ & Capital share of income & 0.35 \\
%         $\delta$ & Capital depreciation rate & $1-(1-0.05)^\frac{80}{S}=0.05$ \\
%         $g_y$ & Growth rate of labor augmenting & $(1+0.03)^\frac{80}{S}-1 = 0.03$ \\[-2mm]
%         & \quad technological progress & \\
%         \hline
%         $T$ & Number of periods to steady state & 160 \\
%         $\nu$ & Dampening parameter for TPI & 0.4 \\
%         \hline\hline
%       \end{tabular}
%       \end{threeparttable}
%     \end{table}

%     Note that the scale parameter $\chi^n_s$ takes on 80 values (one for each model age) that increase with age, representing an increasing disutility of labor that is not modeled anywhere else in the utility function. One might consider this as representing how an hour of labor becomes more costly due to biological reasons related to aging. Such a parametrization helps to fit fact that hours worked decline much more sharply later in life than do hourly earnings.

%     Heterogeneity in the scale parameter multiplying the utility from bequests is useful in having the model generate a distribution of wealth similar to that observed in the data.  Note that without such heterogeneity in this parameter, households at the high end of the earnings distribution in our model would not save as much as their real world counterparts given the deterministic earnings process in our model. They have no precautionary savings motive, only the warm-glow bequest motive for savings.  One can view the assumption of heterogeneous utility weights as not just variation in preference across households, but also as reflecting differences in family size, expectations of income growth, or other variations that are not explicitly modeled here.  We thus allow  $\{\chi^b_j\}_{j=1}^7$ to take on seven values, one for each lifetime income group.


% \newpage
% \section{Characteristics of exogenous population dynamics}\label{AppPopDyn}

%   In this appendix, we detail how we generate the exogenous population dynamics that are inputs to the model described in Section \ref{SecPopDyn}. All output, tests, functions, and computation in this chapter are available in the \href{https://github.com/rickecon/OG-USA/blob/demog/Python/ogusa/demog.py}{\texttt{demog.py}} file.

%   \begin{figure}[htbp]\centering \captionsetup{width=4.0in}
%     \caption{\label{FigPerTime}\textbf{Correspondence of model timing to data timing for model periods of one year}}
%     \fbox{\resizebox{4.0in}{2.0in}{\includegraphics{./images/FigPerTime.pdf}}}
%   \end{figure}

%   Figure \ref{FigPerTime} shows the correspondence between model periods and data periods. Period $s=1$ corresponds to the first year of life between birth and when an household turns one year old. We use this convention to match our model periods to those in the data.


%   \subsection{Nonstationary and stationary population dynamics}\label{AppPopNonStatStat}

%     We define $\omega_{s,t}$ as the number of households of age $s$ alive at time $t$. A measure $\omega_{1,t}$ of households with heterogeneous working ability is born in each period $t$ and live for up to $E+S$ periods, with $S\geq 4$.\footnote{Theoretically, the model works without loss of generality for $S\geq 3$. However, because we are calibrating the ages outside of the economy to be one-fourth of $S$ (e.g., ages 21 to 100 in the economy, and ages 1 to 20 outside of the economy), we need $S$ to be at least 4.} households are termed ``youth'', and do not participate in market activity during ages $1\leq s\leq E$. The households enter the workforce and economy in period $E+1$ and remain in the workforce until they unexpectedly die or live until age $s=E+S$. We model the population with households age $s\leq E$ outside of the workforce and economy in order most closely match the empirical population dynamics.

%     The population of agents of each age in each period $\omega_{s,t}$ evolves according to the following function,
%     \begin{equation}\tag{\ref{EqPopLawofmotion}}
%       \begin{split}
%         \omega_{1,t+1} &= (1 - \rho_0)\sum_{s=1}^{E+S} f_s\omega_{s,t} + i_1\omega_{1,t}\quad\forall t \\
%         \omega_{s+1,t+1} &= (1 - \rho_s)\omega_{s,t} + i_{s+1}\omega_{s+1,t}\quad\forall t\quad\text{and}\quad 1\leq s \leq E+S-1
%       \end{split}
%     \end{equation}
%     where $f_s\geq 0$ is an age-specific fertility rate, $i_s$ is an age-specific net immigration rate, $\rho_s$ is an age specific mortality hazard rate,\footnote{The parameter $\rho_s$ is the probability that a household of age $s$ dies before age $s+1$.} and $\rho_0$ is an infant mortality rate. The total population in the economy $N_t$ at any period is simply the sum of households in the economy, the population growth rate in any period $t$ from the previous period $t-1$ is $g_{n,t}$, $\tilde{N}_t$ is the working age population, and $\tilde{g}_{n,t}$ is the working age population growth rate in any period $t$ from the previous period $t-1$.
%     \begin{equation}\tag{\ref{EqPopN}}
%       N_t\equiv\sum_{s=1}^{E+S} \omega_{s,t} \quad\forall t
%     \end{equation}
%     \begin{equation}\tag{\ref{EqPopGrowth}}
%       g_{n,t+1} \equiv \frac{N_{t+1}}{N_t} - 1 \quad\forall t
%     \end{equation}
%     \begin{equation}\tag{\ref{EqPopNtil}}
%       \tilde{N}_t\equiv\sum_{s=E+1}^{E+S} \omega_{s,t} \quad\forall t
%     \end{equation}
%     \begin{equation}\tag{\ref{EqPopGrowthTil}}
%       \tilde{g}_{n,t+1} \equiv \frac{\tilde{N}_{t+1}}{\tilde{N}_t} - 1 \quad\forall t
%     \end{equation}

%     We can transform the nonstationary equations in \eqref{EqPopLawofmotion} into stationary laws of motion by dividing both sides by the total economically relevant population in the current period $\tilde{N}_t$ and then multiplying the left-hand-side of the equation by $\tilde{N}_{t+1}/\tilde{N}_{t+1}$,
%     \begin{equation}\label{EqPopLawofmotionStat}
%       \begin{split}
%         \hat{\omega}_{1,t+1} &= \frac{(1-\rho_0)\sum_{s=1}^{E+S} f_s\hat{\omega}_{s,t} + i_1\hat{\omega}_{1,t}}{1+\tilde{g}_{n,t+1}}\quad\forall t \\
%         \hat{\omega}_{s+1,t+1} &= \frac{(1 - \rho_s)\hat{\omega}_{s,t} + i_{s+1}\hat{\omega}_{s+1,t}}{1+\tilde{g}_{n,t+1}}\qquad\quad\:\forall t\quad\text{and}\quad 1\leq s \leq E+S-1
%       \end{split}
%     \end{equation}
%     where $\hat{\omega}_{s,t}$ is the percent of the total economically relevant population $\tilde{N}_t$ in age cohort $s$ in period $t$, and $\tilde{g}_{n,t+1}$ is the population growth rate between periods $t$ and $t+1$ defined in \eqref{EqPopGrowthTil}.\footnote{Note in the specification of the stationary laws of motion \eqref{EqPopLawofmotionStat} that $\sum_{s=1}^{E+S}\hat{\omega}_{s,t}>1$ while $\sum_{s=E+1}^{E+S}\hat{\omega}_{s,t}=1$. This is because in the model we only look at the economically relevant population $\hat{\omega}_{s,t}$ for $E+1\leq s\leq E+S$.}


%   \subsection{Fertility rates}\label{AppPopFert}

%     In this model, we assume that the fertility rates for each age cohort $f_s$ are constant across time. However, this assumption is conceptually straightforward to relax. Our data for U.S. fertility rates by age come from \citet[Table 3, p. 18]{MartinEtAl:2015} National Vital Statistics Report, which is final fertility rate data for 2013. Figure \ref{FigFertRates} shows the fertility-rate data and the estimated average fertility rates for $E+S=100$.

%     \begin{figure}[htbp]\centering \captionsetup{width=4.0in}
%       \caption{\label{FigFertRates}\textbf{Fertility rates by age ($f_s$) for $E+S=100$}}
%       \fbox{\resizebox{4.0in}{3.0in}{\includegraphics{./images/fert_rates.png}}}
%     \end{figure}

%     The large blue circles are the 2013 U.S. fertility rate data from \citet{MartinEtAl:2015}. These are 9 fertility rates $[0.3, 12.3, 47.1, 80.7, 105.5, 98.0, 49.3, 10.4, 0.8]$ that correspond to the midpoint ages of the following age (in years) bins $[10-14, 15-17, 18-19, 20-24, 25-29, 30-34, 35-39, 40-44, 45-49]$. In order to get our cubic spline interpolating function to fit better at the endpoints we added to fertility rates of zero to ages 9 and 10, and we added two fertility rates of zero to ages 55 and 56. The blue line in Figure \ref{FigFertRates} shows the cubic spline interpolated function of the data.

%     The red diamonds in Figure \ref{FigFertRates} are the average fertility rate in age bins spanning households born at the beginning of period 1 (time = 0) and dying at the end of their 100th year. Let the total number of model years that a household lives be \texttt{totpers}, which is just $E+S\leq 100$. Then the span from 0 to 100 is divided up into \texttt{totpers} bins of equal length. We calculate the average fertility rate in each of the \texttt{totpers} model-period bins as the average population-weighted fertility rate in that span. The red diamonds in Figure \ref{FigFertRates} are the average fertility rates displayed at the midpoint in each of the \texttt{totpers} model-period bins.


%   \subsection{Mortality rates}\label{AppPopMort}

%     The mortality rates in our model $\rho_s$ are a one-period hazard rate and represent the probability of dying within one year, given that an household is alive at the beginning of period $s$. We assume that the mortality rates for each age cohort $\rho_s$ are constant across time. The infant mortality rate of $\rho_0=0.00587$ comes from the 2015 U.S. CIA World Factbook. Our data for U.S. mortality rates by age come from the Actuarial Life Tables of the U.S. Social Security Administration \citep[see][]{SocSec:2015}, from which the most recent mortality rate data is for 2011. Figure \ref{FigMortRates} shows the mortality rate data and the corresponding model-period mortality rates for $E+S=100$.

%     \begin{figure}[htbp]\centering \captionsetup{width=4.0in}
%       \caption{\label{FigMortRates}\textbf{Mortality rates by age ($\rho_s$) for $E+S=100$}}
%       \fbox{\resizebox{4.0in}{3.0in}{\includegraphics{./images/mort_rates.png}}}
%     \end{figure}

%     The mortality rates in Figure \ref{FigMortRates} are a population-weighted average of the male and female mortality rates reported in \citet{SocSec:2015}. Figure \ref{FigMortRates} also shows that the data provide mortality rates for ages up to 111-years-old. We truncate the maximum age in years in our model to 100-years old. In addition, we constrain the mortality rate to be 1.0 or 100 percent at the maximum age of 100.


%   \subsection{Immigration rates}\label{AppPopImm}

%     Because of the difficulty in getting accurate immigration rate data by age, we estimate the immigration rates by age in our model $i_s$ as the average residual that reconciles the current-period population distribution with next period's population distribution given fertility rates $f_s$ and mortality rates $\rho_s$. Solving equations \eqref{EqPopLawofmotion} for the immigration rate $i_s$ gives the following characterization of the immigration rates in given population levels in any two consecutive periods $\omega_{s,t}$ and $\omega_{s,t+1}$ and the fertility rates $f_s$ and mortality rates $\rho_s$.

%     \begin{equation}\label{EqPopImmRates}
%       \begin{split}
%         i_1 &= \frac{\omega_{1,t+1} - (1 - \rho_0)\sum_{s=1}^{E+S}f_s\omega_{s,t}}{\omega_{1,t}}\quad\forall t \\
%         i_{s+1} &= \frac{\omega_{s+1,t+1} - (1 - \rho_s)\omega_{s,t}}{\omega_{s+1,t}}\qquad\qquad\forall t\quad\text{and}\quad 1\leq s \leq E+S-1
%       \end{split}
%     \end{equation}

%     \begin{figure}[htbp]\centering \captionsetup{width=4.0in}
%       \caption{\label{FigImmRates}\textbf{Immigration rates by age ($i_s$), residual, $E+S=100$}}
%       \fbox{\resizebox{4.0in}{3.0in}{\includegraphics{./images/imm_rates_orig.png}}}
%     \end{figure}

%     We calculate our immigration rates for three different consecutive-year-periods of population distribution data (2010 through 2013). Our four years of population distribution by age data come from \citet{Census:2015}. The immigration rates $i_s$ that we use in our model are the the residuals described in \eqref{EqPopImmRates} averaged across the three periods. Figure \ref{FigImmRates} shows the estimated immigration rates generated from our \texttt{get\_imm\_resid()} function for $E+S=100$ and given the fertility rates from Section \ref{AppPopFert} and the mortality rates from Section \ref{AppPopMort}.


%   \subsection{Population steady state and transition}\label{AppPopSStrans}

%     This model requires information about mortality rates $\rho_s$ in order to solve for the household's problem each period. It also requires the steady-state stationary population distribution $\bar{\omega}_{s}$ as well as the full transition path of the stationary population distribution $\hat{\omega}_{s,t}$ from the current state to the steady-state. To solve for the steady-state and the transition path of the stationary population distribution, we write the stationary population dynamic equations from \eqref{EqPopLawofmotionStat} in matrix form.
%     \begin{equation}\label{EqPopLOMstatmat}
%       \begin{split}
%         & \begin{bmatrix}
%           \hat{\omega}_{1,t+1} \\ \hat{\omega}_{2,t+1} \\ \hat{\omega}_{2,t+1} \\ \vdots \\ \hat{\omega}_{E+S-1,t+1} \\ \hat{\omega}_{E+S,t+1}
%         \end{bmatrix}= \frac{1}{1 + g_{n,t+1}} \times ... \\
%         & \begin{bmatrix}
%           (1-\rho_0)f_1+i_1 & (1-\rho_0)f_2 & (1-\rho_0)f_3 & \hdots & (1-\rho_0)f_{E+S-1} & (1-\rho_0)f_{E+S} \\
%           1-\rho_1 & i_2 & 0 & \hdots & 0 & 0 \\
%           0 & 1-\rho_2 & i_3 & \hdots & 0 & 0 \\
%           \vdots & \vdots & \vdots & \ddots & \vdots & \vdots \\
%           0 & 0 & 0 & \hdots & i_{E+S-1} & 0 \\
%           0 & 0 & 0 & \hdots & 1-\rho_{E+S-1} & i_{E+S}
%         \end{bmatrix}
%         \begin{bmatrix}
%           \hat{\omega}_{1,t} \\ \hat{\omega}_{2,t} \\ \hat{\omega}_{2,t} \\ \vdots \\ \hat{\omega}_{E+S-1,t} \\ \hat{\omega}_{E+S,t}
%         \end{bmatrix}
%       \end{split}
%     \end{equation}
%     We can write system \eqref{EqPopLOMstatmat} more simply in the following way.
%     \begin{equation}\label{EqPopLOMstatmat2}
%       \bm{\hat{\omega}}_{t+1} = \frac{1}{1+g_{n,t+1}}\bm{\Omega}\bm{\hat{\omega}}_t \quad\forall t
%     \end{equation}
%     The stationary steady-state population distribution $\bm{\bar{\omega}}$ is the eigenvector $\bm{\omega}$ with eigenvalue $(1+\bar{g}_n)$ of the matrix $\bm{\Omega}$ that satisfies the following version of \eqref{EqPopLOMstatmat2}.
%     \begin{equation}\label{EqPopLOMss}
%       (1+\bar{g}_n)\bm{\bar{\omega}} = \bm{\Omega}\bm{\bar{\omega}}
%     \end{equation}

%     \begin{proposition}
%       If the age $s=1$ immigration rate is $i_1>-(1-\rho_0)f_1$ and the other immigration rates are strictly positive $i_s>0$ for all $s\geq 2$ such that all elements of $\bm{\Omega}$ are nonnegative, then there exists a unique positive real eigenvector $\bm{\bar{\omega}}$ of the matrix $\bm{\Omega}$, and it is a stable equilibrium.
%     \end{proposition}

%     \begin{proof}
%       First, note that the matrix $\bm{\Omega}$ is square and non-negative.  This is enough for a general version of the Perron-Frobenius Theorem to state that a positive real eigenvector exists with a positive real eigenvalue. This is not yet enough for uniqueness. For it to be unique by a version of the Perron-Fobenius Theorem, we need to know that the matrix is irreducible. This can be easily shown. The matrix is of the form
%       $$\bm{\Omega} =
%       \begin{bmatrix}
%         * & *  & * & \hdots & * & * & *\\
%         * & * & 0 & \hdots & 0 & 0 & 0 \\
%         0 & * & * & \hdots & 0 & 0 & 0 \\
%         \vdots & \vdots & \vdots & \ddots & \vdots & \vdots & \vdots \\
%         0 & 0 & 0 & \hdots & *  & * & 0 \\
%         0 & 0 & 0 & \hdots & 0 & * & *
%       \end{bmatrix}
%       $$
%       Where each * is strictly positive. It is clear to see that taking powers of the matrix causes the sub-diagonal positive elements to be moved down a row and another row of positive entries is added at the top. None of these go to zero since the elements were all non-negative to begin with.
%       $$\bm{\Omega}^2 =
%       \begin{bmatrix}
%         * & *  & * & \hdots & * & * & *\\
%         * & * & * & \hdots & * & * & * \\
%         0 & * & * & \hdots & 0 & 0 & 0 \\
%         \vdots & \vdots & \vdots & \ddots & \vdots & \vdots & \vdots \\
%         0 & 0 & 0 & \hdots & *  & * & 0 \\
%         0 & 0 & 0 & \hdots & 0 & * & *
%       \end{bmatrix}; ~~~
%       \bm{\Omega}^{S+E-1} =
%       \begin{bmatrix}
%         * & *  & * & \hdots & * & * & *\\
%         * & * & * & \hdots & * & * & * \\
%         * & * & * & \hdots & * & * & * \\
%         \vdots & \vdots & \vdots & \ddots & \vdots & \vdots & \vdots \\
%         * & * & * & \hdots & *  & * & * \\
%         0 & 0 & 0 & \hdots & 0 & * & *
%       \end{bmatrix}
%       $$
%       $$\bm{\Omega}^{S+E} =
%       \begin{bmatrix}
%         * & *  & * & \hdots & * & * & *\\
%         * & * & * & \hdots & * & * & * \\
%         * & * & * & \hdots & * & * & * \\
%         \vdots & \vdots & \vdots & \ddots & \vdots & \vdots & \vdots \\
%         * & * & * & \hdots & * & * & * \\
%         * & * & * & \hdots & * & * & *
%       \end{bmatrix}
%       $$
%       Existence of an $m \in \mathbb N $ such that $\left(\bf\Omega^m\right)_{ij} \neq 0 ~~ ( > 0)$ is one of the definitions of an irreducible (primitive) matrix. It is equivalent to saying that the directed graph associated with the matrix is strongly connected. Now the Perron-Frobenius Theorem for irreducible matrices gives us that the equilibrium vector is unique.

%       We also know from that theorem that the eigenvalue associated with the positive real eigenvector will be real and positive. This eigenvalue, $p$, is the Perron eigenvalue and it is the steady state population growth rate of the model. By the PF Theorem for irreducible matrices, $| \lambda_i | \leq p$ for all eigenvalues $\lambda_i$ and there will be exactly $h$ eigenvalues that are equal, where $h$ is the period of the matrix. Since our matrix $\bf\Omega$ is aperiodic, the steady state growth rate is the unique largest eigenvalue in magnitude. This implies that almost all initial vectors will converge to this eigenvector under iteration.
%     \end{proof}

%     For a full treatment and proof of the Perron-Frobenius Theorem, see \citet{Suzumura:1983}. Because the population growth process is exogenous to the model, we calibrate it to annual age data for age years $s=1$ to $s=100$.

%     Figure \ref{FigOrigVsFixSSpop} shows the steady-state population distribution $\bm{\bar{\omega}}$ and the population distribution after 120 periods $\bm{\hat{\omega}}_{120}$. Although the two distributions look very close to each other, they are not exactly the same.

%     \begin{figure}[htbp]\centering \captionsetup{width=4.0in}
%       \caption{\label{FigOrigVsFixSSpop}\textbf{Theoretical steady-state population distribution vs. population distribution at period $t=120$}}
%       \fbox{\resizebox{4.0in}{3.0in}{\includegraphics{./images/OrigVsFixSSpop.png}}}
%     \end{figure}

%     Further, we find that the maximum absolute difference between the population levels $\hat{\omega}_{s,t}$ and $\hat{\omega}_{s,t+1}$ was $1.3852\times 10^{-5}$ after 160 periods. That is to say, that after 160 periods, given the estimated mortality, fertility, and immigration rates, the population has not achieved its steady state. For convergence in our solution method over a reasonable time horizon, we want the population to reach a stationary distribution after $T$ periods. To do this, we artificially impose that the population distribution in period $t=120$ is the steady-state. As can be seen from Figure \ref{FigOrigVsFixSSpop}, this assumption is not very restrictive. Figure \ref{FigImmRateChg} shows the change in immigration rates that would make the period $t=120$ population distribution equal be the steady-state. The maximum absolute difference between any two corresponding immigration rates in Figure \ref{FigImmRateChg} is 0.0028.

%     \begin{figure}[htbp]\centering \captionsetup{width=4.0in}
%       \caption{\label{FigImmRateChg}\textbf{Original immigration rates vs. adjusted immigration rates to make fixed steady-state population distribution}}
%       \fbox{\resizebox{4.0in}{3.0in}{\includegraphics{./images/OrigVsAdjImm.png}}}
%     \end{figure}

%     The most recent year of population data come from \citet{Census:2015} population estimates for both sexes for 2013. We those data and use the population transition matrix \eqref{EqPopLOMstatmat2} to age it to the current model year of 2015. We then use \eqref{EqPopLOMstatmat2} to generate the transition path of the population distribution over the time period of the model. Figure \ref{FigPopDistPath} shows the progression from the 2013 population data to the fixed steady-state at period $t=120$. The time path of the growth rate of the economically active population $\tilde{g}_{n,t}$ is shown in Figure \ref{FigGrowthPath}.

%     \begin{figure}[htbp]\centering \captionsetup{width=4.0in}
%       \caption{\label{FigPopDistPath}\textbf{Stationary population distribution at periods along transition path}}
%       \fbox{\resizebox{4.0in}{3.0in}{\includegraphics{./images/PopDistPath.png}}}
%     \end{figure}

%     \begin{figure}[htbp]\centering \captionsetup{width=4.0in}
%       \caption{\label{FigGrowthPath}\textbf{Time path of the population growth rate $\tilde{g}_{n,t}$}}
%       \fbox{\resizebox{4.0in}{3.0in}{\includegraphics{./images/GrowthPath.png}}}
%     \end{figure}
%     \clearpage


% \newpage
% \section{Solving for stationary steady-state equilibrium}\label{AppSSsolve}

%   \setcounter{equation}{0}
%   \renewcommand\theenumi{\arabic{enumi}}
%   \renewcommand\theenumii{\alph{enumii}}
%   \renewcommand\theenumiii{\roman{enumiii}}

%   This section describes the solution method for the stationary steady-state equilibrium described in Definition \ref{DefEquilSS}. The steady-state is characterized by $2JS$ equations and $2JS$ unknowns. However, because some of the other equations cannot be solved for analytically and substituted into the Euler equations, we must take a two-stage approach to the equilibrium solution. We first make a guess at steady-state wage $\bar{w}$, interest rate $\bar{r}$, lump-sum transfer $\bar{T}^H$, and income multiplier $factor$. Then, given those four aggregate variables, we can solve for the second-stage household decisions of steady-state savings $\bar{b}_{j,s}$ and labor supply $\bar{n}_{j,s}$.

%   \begin{enumerate}
%     \item Use the techniques in Appendix \ref{AppPopDyn} to solve for the steady-state population distribution vector $\bm{\bar{\omega}}$ of the exogenous population process.
%     \item Choose an initial guess for the values of the steady-state wage $\bar{w}$, interest rate $\bar{r}$, lump-sum transfer $\bar{T}^H$, and income multiplier $factor$.
%     \item Given guesses for $\bar{w}$, $\bar{r}$, $\bar{T}^H$, and $factor$, solve for the steady-state household savings $\bar{b}_{j,s}$ and labor supply $\bar{n}_{j,s}$ decisions using $2JS$ equations \eqref{EqEulerLabStat}, \eqref{EqEulerSavStat}.
%       \begin{itemize}
%         \item A good first guess for $\bar{b}_{j,s}$ and $\bar{n}_{j,s}$ is a number close to but less than $\tilde{l}$ for all the $\bar{n}_{j,s}$ and to choose some small positive number for $\bar{b}_{j,s}$ that is small enough to be less than the minimum income that an household might have $\bar{w}e_{j,s}\bar{n}_{j,s}$.
%         \item Make sure that all of the $2JS$ Euler errors is sufficiently close to zero to constitute a solution.
%       \end{itemize}
%     \item Given the solutions $\bar{b}_{j,s}$ and $\bar{n}_{j,s}$ from step (3), make sure that the four characterizing equations for $\bar{w}$, $\bar{r}$, $\bar{T}^H$, and $factor$ are solved. These characterizing equations are the zero equations corresponding to the steady-state versions of \eqref{EqFOCwageStat}, \eqref{EqFOCrate}, \eqref{EqGovtBCstat}, and \eqref{EqIncFactor}.
%       \begin{align}
%         \bar{w} - (1-\alpha)\frac{\bar{Y}}{\bar{L}} &= 0 \label{EqFOCwageSS0} \\
%         \bar{r} - \alpha\frac{\bar{Y}}{\bar{K}} + \delta &= 0 \label{EqFOCrateSS0} \\
%         \bar{T}^H - \sum_s \sum_j \bar{\omega}_s\lambda_j \bar{T}_{s} &= 0 \label{EqGovtBCSS0} \\
%         factor\sum_s \sum_j \bar{\omega}_s\lambda_j\left(\bar{w}e_{j,s}\bar{n}_{j,s} + \bar{r}\bar{b}_{j,s}\right) - \left(\text{data avg. income}\right) &= 0 \label{EqIncFactorSS0}
%       \end{align}
%     \item Iterate on guesses for outer loop values of $\bar{w}$, $\bar{r}$, $\bar{T}^H$, and $factor$ until the Euler equations from step (3) and the characterizing equations from step (4) are all solved.
%   \end{enumerate}


% \newpage
% \section{Solving for stationary non-steady-state equilibrium by time path iteration}\label{AppNonSSsolve}

%   \setcounter{equation}{0}

%   This section describes the solution to the non-steady-state transition path equilibrium of the model described in Definition \ref{DefEquilNonSS} and outlines the time path iteration (TPI) method of \citet{AuerbachKotlikoff:1987} for solving for this equilibrium. The following are the steps for computing a stationary non-steady-state equilibrium time path for the economy.
%   \begin{enumerate}
%     \item Input all initial parameters. See Table \ref{TabExogVars}.
%       \begin{enumerate}
%         \item The value for $T$ at which the non-steady-state transition path should have converged to the steady state should be at least as large as the number of periods it takes the population to reach its steady state $\bm{\bar{\omega}}$ as described in Appendix \ref{AppPopDyn}.
%       \end{enumerate}
%     \item Choose an initial distribution of savings and intended bequests $\bm{\hat{\Gamma}}_1$ and then calculate the initial state of the stationarized aggregate capital stock $\hat{K}_1$ and total bequests received $\hat{BQ}_{j,1}$ consistent with $\bm{\hat{\Gamma}}_1$ according to \eqref{EqMktClrCapStat} and \eqref{EqTotBeqStat}.
%       \begin{enumerate}
%         \item Note that you must have the population weights from the previous period $\hat{\omega}_{s,0}$ and the growth rate between period 0 and period 1 $\tilde{g}_{n,1}$ to calculate $\hat{BQ}_{j,1}$.
%       \end{enumerate}
%     \item Conjecture transition paths for the stationarized wage $\bm{\hat{w}}^1=\{\hat{w}^1_t\}_{t=1}^\infty$, stationarized interest rate $\bm{r}^1=\{r^1_t\}_{t=1}^\infty$, total bequests received $\bm{\hat{BQ}}_j^1=\{\hat{BQ}^{1}_{j,t}\}_{t=1}^\infty$ for each household type $j$, and the lump-sum transfer from the government $\bm{\hat{T}^{H,1}}=\{\hat{T}^{H,1}_t\}_{t=1}^\infty$. The only requirements are that $\hat{K}^i_1$ and $\hat{BQ}^i_{j,1}$ are functions of the initial distribution of savings $\bm{\hat{\Gamma}}_1$ for all iterations $i$ in your initial state and that the time paths of $\bm{\hat{w}}^i$, $\bm{r}^i$, $\bm{\hat{BQ}}_j^i$, and $\bm{\hat{T}^{H,i}}$ equal their respective steady-state values for all $t\geq T$.
%       \begin{enumerate}
%         \item Initial guesses for $\hat{w}_1$ and $r_1$ can be disciplined a little bit by whether $\hat{K}_1$ is greater than or less than $\bar{K}$. If $\hat{K}_1 > \bar{K}$, then choose $\hat{w}_1 > \bar{w}$ and $r_1 < \bar{r}$. If $\hat{K}_1 < \bar{K}$, then choose $\hat{w}_1 < \bar{w}$ and $r_1 > \bar{r}$.
%       \end{enumerate}
%     \item With the conjectured transition paths $\bm{\hat{w}}^i$, $\bm{r}^i$, $\bm{\hat{BQ}}_j^i$, and $\bm{\hat{T}^{H,i}}$, one can solve for the lifetime labor and savings decisions for each household in the model who will be alive between periods $t=1$ and $T$. Each household's lifetime decisions can be solved independently using the systems of $2S$ Euler equations of the form \eqref{EqEulerLabStat}, \eqref{EqEulerSavStat}, and \eqref{EqEulerSavEpSstat}.
%       \begin{enumerate}
%         \item Make sure all the Euler errors for both the savings and labor supply decisions are sufficiently close to zero in order to ensure that the household equilibrium is being solved.
%       \end{enumerate}
%     \item Use the implied distribution of savings and labor supply in each period to compute the new implied time paths for the wage $\bm{\hat{w}}^{i'} = \{\hat{w}_1^i,\hat{w}_2^{i'},...\hat{w}_T^{i'}\}$, interest rate $\bm{r}^{i'} = \{r_1^i,r_2^{i'},...r_T^{i'}\}$, total bequests received $\bm{\hat{BQ}}_j^{i'} = \{\hat{BQ}_{j,1}^i,\hat{BQ}_{j,2}^{i'},...\hat{BQ}_{j,T}^{i'}\}$ for each ability group $j$, and lump-sum transfer from the government $\bm{\hat{T}^{H,i'}} = \{\hat{T}_1^{H,i'},\hat{T}_2^{H,i'},...\hat{T}_T^{H,i'}\}$.
%     \item Check the distance between the two sets time paths.
%       \begin{equation*}
%         \norm{\Bigl[\bm{\hat{w}}^{i'}, \bm{r}^{i'},\bigl\{\bm{\hat{BQ}}_j^{i'}\bigr\}_{j=1}^J, \bm{\hat{T}^{H,i'}}\Bigr] - \Bigl[\bm{\hat{w}}^{i}, \bm{r}^{i},\bigl\{\bm{\hat{BQ}}_j^{i}\bigr\}_{j=1}^J, \bm{\hat{T}^{H,i}}\Bigr]}
%       \end{equation*}
%       \begin{enumerate}
%         \item If the distance between the initial time paths and the implied time paths is less-than-or-equal-to some convergence criterion $\ve>0$, then the fixed point has been achieved and the equilibrium time path has been found.
%         \item If the distance between the initial time paths and the implied time paths is greater than some convergence criterion $\norm{\cdot}>\ve$, then update the guess for the time paths and repeat steps (4) through (6) until a fixed point is reached.
%       \end{enumerate}
%   \end{enumerate}

%   Figures \ref{FigKpathTPI}, \ref{FigIpathTPI}, and \ref{FigLpathTPI} show the equilibrium time paths of the aggregate capital stock $K_t$, aggregate investment $I_t$, and aggregate labor supply $L_t$ for the calibration of the model in this paper.

%   \begin{figure}[htb]\centering \captionsetup{width=4.0in}
%     \caption{\label{FigKpathTPI}\textbf{Equilibrium time path of aggregate capital $K_t$ for $S=80$ and $J=7$ in baseline model}}
%     \fbox{\resizebox{4.0in}{3.0in}{\includegraphics{./images/Kpath.png}}}
%   \end{figure}

%   \begin{figure}[htb]\centering \captionsetup{width=4.0in}
%     \caption{\label{FigIpathTPI}\textbf{Equilibrium time path of aggregate investment $I_t$ for $S=80$ and $J=7$ in baseline model}}
%     \fbox{\resizebox{4.0in}{3.0in}{\includegraphics{./images/Ipath.png}}}
%   \end{figure}

%   \begin{figure}[htb]\centering \captionsetup{width=4.0in}
%     \caption{\label{FigLpathTPI}\textbf{Equilibrium time path of aggregate labor $L_t$ for $S=80$ and $J=7$ in baseline model}}
%     \fbox{\resizebox{4.0in}{3.0in}{\includegraphics{./images/Lpath.png}}}
%   \end{figure}

%   \clearpage

% \end{spacing}

\end{document}
